\documentclass[10pt]{article}

\usepackage[margin=1in]{geometry} 
\usepackage{amsmath,amsthm,amssymb, graphicx, multicol, array}
\usepackage{tikz-cd} 

\newcommand{\N}{\mathbb{N}}
\newcommand{\Z}{\mathbb{Z}}
\newcommand{\Q}{\mathbb{Q}}
\newcommand{\R}{\mathbb{R}}
\newcommand{\C}{\mathbb{C}}
\newcommand{\Gal}{\text{Gal}}

\newenvironment{problem}[2][Problem]{\begin{trivlist}
		\item[\hskip \labelsep {\bfseries #1}\hskip \labelsep {\bfseries #2.}]}{\end{trivlist}}

\newenvironment{answer}[2][Answer]{\begin{trivlist}
		\item[\hskip \labelsep {\bfseries #1}\hskip \labelsep {\bfseries #2.}]}{\end{trivlist}}

\newenvironment{theorem}[2][Theorem]{\begin{trivlist}
		\item[\hskip \labelsep {\bfseries #1}\hskip \labelsep {\bfseries #2.}]}{\end{trivlist}}

\newtheorem{thm}{Theorem}
\newtheorem{defn}{Definition}
\newtheorem{conv}{Convention}
\newtheorem{rem}{Remark}
\newtheorem{lem}{Lemma}
\newtheorem{cor}{Corollary}

\begin{document}
	
	\title{Exercises from Dummit and Foote Chapter 14 on Galois Theory}
	\author{Wesley Basener}
	\maketitle
	
	\begin{problem}{1.1}
		\begin{itemize}
			\item[\textbf{(a)}]
				Show that if the field $K$ is generated over $F$ by the elements $\alpha_1, ..., \alpha_n$ then an
				automorphism $\sigma$ of $K$ fixing $F$ is uniquely determined by $\sigma (\alpha_1), ..., \sigma (\alpha_n )$. In
				particular show that an automorphism fixes $K$ if and only if it fixes a set of generators
				for $K$.
			\item[\textbf{(b)}]
				Let $G \leq = \Gal(K/F)$ b e a subgroup of the Galois group of the extension $K/F$ and
				suppose $\sigma_1, ..., \sigma_k$ are generators for $G$. Show that the subfield $E/F$ is fixed by $G$ if
				and only if it is fixed by the generators $\sigma_1, ..., \sigma_k$.
		\end{itemize}
		\begin{proof}{(a)}
			  Let $\sigma$ be any automorphism on $K$ fixing $F$. Then, for any $k = a_0 + a_1\alpha_1 + ... + a_n\alpha_n$ in $K$, $\sigma(k) = \sigma(a_0) + \sigma(a_1)\sigma(\alpha_1) + ... + \sigma(a_n)\sigma(\alpha_n)$. Using the fact that $\sigma$ fixes $F$, we have $\sigma(k) = a_0 + a_1\sigma(\alpha_1) + ... + a_n\sigma(\alpha_n)$. Hence the image of any $k \in K$ on $\sigma$ is uniquely determined by $\sigma (\alpha_1), ..., \sigma (\alpha_n )$.
			  From this, it is obvious that $\sigma$ fixes $K$ if it fixes the generators for $K$.
		\end{proof}
		\begin{proof}[(b)]
			Denote the generators of $E$ over $F$ by $\alpha_1, ..., \alpha_m$. Suppose $G$ fixes $E/F$. From part (a), this is true if and only if $\sigma_i(\alpha_j) = \alpha_j$ for all $i \in [1,k], j \in [1,m]$. Hence, any element of a $E/F$ is fixed by any element of $G$.
		\end{proof}
	\end{problem}
	
	\begin{problem}{1.3}
		Determine the fixed field of complex conjugation on $\C$.
		\begin{proof}
			Complex conjugation is the function $\sigma : a+bi \mapsto a-bi$, which obviously fixes $a$. Hence, the fixed field of complex conjugation is $\R$ the real numbers.
		\end{proof}
	\end{problem}
	
	\begin{problem}{1.5}
		Determine the automorphisms of the extension $\Q(\sqrt[4]{2})/\Q(\sqrt{2})$ explicitly.
		\begin{proof}
			There is only one basis element to this extension, namely $\sqrt[4]{2}$. Since $-\sqrt[4]{2} \not = \sqrt[4]{2}$, the automorphism $\sigma : a + b\sqrt[4]{2} \mapsto a - b\sqrt[4]{2}$ is not the identity. Hence, the automorphisms of this extension are $\{1, \sigma\}$.
		\end{proof}
	\end{problem}
	
	\begin{problem}{1.7}
		This problem determines $\text{Aut}(\R/\Q)$.
		\begin{itemize}
			\item[\textbf{(a)}]
				Prove that any $\sigma \in \text{Aut}(\R/\Q)$ takes squares to squares and takes positive reals to positive reals. Conclude that $a < b$ implies $\sigma a < \sigma b$ for every $a, b \in \R$ Conclude that $a<b$ implies $\sigma a<\sigma b$ for every $a,b \in \R$.
			\item[\textbf{(b)}]
				Prove that $-\frac{1}{m} < a - b < \frac{1}{m}$ implies $-\frac{1}{m} < \sigma a - \sigma b < \frac{1}{m}$ for every positive integer $m$. Conclude that $\sigma$ is a continuous map on $\R$.
			\item[\textbf{(c)}]
				Prove that any continuous map on $\R$ which is the identity on $\Q$ is the identity map, hence $\text{Aut}(\R/\Q)=1$. 
		\end{itemize}
		\begin{proof}{(a)}
			Let $\sigma$ be an automorphism on $\R/\Q$. Suppose $x$ is a real square. Then, $x = p^2$ for real number $p$. Hence, we have $\sigma(x) = \sigma(p^2) = \sigma(p)\sigma(p)$. Thus, $\sigma$ sends squares to squares. 
			
			Let $y$ be any positive real number. Since $y$ is positive, $\sqrt{y}$ is real. From the first part of this proof, we know that $\sigma(y) = \sigma(\sqrt{y}\sqrt{y}) = q^2$ for some real number $q$. Since we are limited to the real numbers, $q^2$ is positive. Hence, $\sigma(y)$ is positive.
			
			For any $a,b \in \R$, $a<b$ implies $0 < b-a$. Hence, from the prior paragraph, $0 < \sigma(b) - \sigma(a)$. Adding $\sigma(a)$ to both sides yields $\sigma(a) < \sigma(b)$. Note that setting $b=0$ proves that $\sigma$ sends negatives to negatives.
		\end{proof}
		\begin{proof}[(b)]
			Suppose $a,b$ are real numbers such that $-\frac{1}{m} < a - b < \frac{1}{m}$ for some positive integer $m$.
			
			Since $m$ is an integer, it can be rewritten as $\Sigma_{i=0}^m 1$. Hence, $\sigma(m) = \Sigma_{i=0}^m \sigma(1) = \Sigma_{i=0}^m 1 = m$.
			
			We can rewrite the above inequality as $-1 < m(a-b) < 1$. Which is the same as having $m(a-b) - 1$ is negative and $m(a-b) + 1$ is positive. From part (a), we know $\sigma$ sends positives to positives and negatives to negatives. Hence, $\sigma(m(a-b) - 1) = m(\sigma(a) - \sigma(b)) - 1$ is negative and $\sigma(m(a-b) - 1) = m(\sigma(a) - \sigma(b)) + 1$ is positive. Which of course implies $-\frac{1}{m} < \sigma(a) - \sigma(b) < \frac{1}{m}$
			
			To show that $\sigma$ is continuous, let $x$ be any real number and let $\epsilon > 0$. We can find a natural number $N$ such that $\frac{1}{N} < \epsilon$. Then, for any $x_0$ such that $|x-x_0| < \frac{1}{N}$, we have $|\sigma(x) - \sigma(x_0)| < \frac{1}{N} < \epsilon$. Hence, $\sigma$ is continuous.
		\end{proof}
		\begin{proof}[(c)]
			Suppose $\sigma \in \text{Aut}(\R/\Q)$ fixes $\Q$. Let $x$ be any real number. Then by the density of the rationals in $\R$, for any $\epsilon > 0$, there exists some $q \in \Q$ such that $|x-q| < \epsilon$ Hence, $|\sigma(x-q)| = |\sigma(x)-q| < \epsilon$ which is only possible if $\sigma(x) = x$. Thus, any such $\sigma$ mus be the identity function. Therefore, $\text{Aut}(\R/\Q)=1$.
		\end{proof}
	\end{problem}
	
	\begin{problem}{1.9}
		Determine the fixed field of the automorphism $t \mapsto t + 1$ of $k(t)$.
		\begin{proof}
			Any element of $k(t)$ will have the form $\frac{\Sigma a_i t^i}{\Sigma b_i t^i}$ with $\gcd(\Sigma a_i x^i, \Sigma b_i x^i) = 1$. 
			Suppose we have an element such that $\frac{\Sigma a_i (t+1)^i}{\Sigma b_i (t+1)^i} = \frac{\Sigma a_i t^i}{\Sigma b_i t^i}$. Then, $\frac{\Sigma a_i (t+1)^i}{\Sigma b_i (t+1)^i} - \frac{\Sigma a_i t^i}{\Sigma b_i t^i} = 0$ and since both fractions remain irreducible, we would have $\Sigma b_i (t+1)^i = \Sigma b_i t^i$. Thus, we would also have $\Sigma a_i (t+1)^i = \Sigma a_i (t)^i$. Hence, the fixed field of $k(t)$ is precisely the set of rational functions whose numerators and denominators are both fixed by the automorphism.
			
			//TODO: finish this proof.
		\end{proof}
	\end{problem}
	
	\begin{problem}{2.1}
		Determine the minimal polynomial over $\Q$ for the element .
		\begin{proof}
			We have that $\Q(\sqrt{2}+\sqrt{5})$ is a subfield of $\Q(\sqrt{2}, \sqrt{5})$, which is the splitting field of $(x^2-2)(x^2-5)$. Since this polynomial is separable, $\Q(\sqrt{2}, \sqrt{5})$ is Galois.
			
			We can therefore find the other roots of the minimal polynomial of $\Q(\sqrt{2}+\sqrt{5})$ by considering the action of $\text{Aut}(\Q/\Q(\sqrt{2}, \sqrt{5})$ on $\sqrt{2}+\sqrt{5}$. This yields $\pm\sqrt{2} \pm\sqrt{5}$, which are indeed distinct. 
			
			Hence, the minimal polynomial of $\sqrt{2}+\sqrt{5}$ is $(x - \sqrt{2}+\sqrt{5})(x + \sqrt{2}+\sqrt{5})(x - \sqrt{2}-\sqrt{5})(x + \sqrt{2}-\sqrt{5})$ which multiplies to $x^4 - 14x^2 + 9$.
			
			
			\rem
				The inverse of $\sqrt{2}+\sqrt{5}$ on $\Q(\sqrt{2}+\sqrt{5})$ is $\frac{\sqrt{2}-\sqrt{5}}{-3}$. Hence, the field $\Q(\sqrt{2}+\sqrt{5})$ contains $\sqrt{5}$ and $\sqrt{2}$. Given that $\Q(\sqrt{2}+\sqrt{5})$ is a subfield of $\Q(\sqrt{2}, \sqrt{5})$, we have that $\Q(\sqrt{2} + \sqrt{5}) = \Q(\sqrt{2}, \sqrt{5})$.
			
				From this, I initially though that the minimal polynomial of $\sqrt{2}+\sqrt{5}$ would be the same as the minimal polynomial with roots $\sqrt{2}$ and $\sqrt{5}$. But this is obviously not the case since $(\sqrt{2}+\sqrt{5})$ is not a root of $(x^2 - 5)(x^2 - 2)$.
				
				This is a case of being disillusioned of unjustified assumptions. Just because $F(a) = F(b,c)$, does not mean that the minimal polynomial of $a$ and the minimal polynomial with roots $b,c$ are the same. In this case, $(x^2 - 5)(x^2 - 2)$ is not reducible, so it is not a minimal polynomial for anything.
		\end{proof}
	\end{problem}
	
	\begin{problem}{2.3}
		Determine the Galois group of $(x^2 - 2)(x^2 - 3)(x^2 - 5)$. Determine all the subfields of the splitting field of this polynomial.
		\begin{proof}
			This polynomial is separable with roots $\pm\sqrt{2}$, $\pm\sqrt{3}$, and $\pm\sqrt{5}$. Hence, its splitting field $K = \Q(\sqrt{2},\sqrt{3},\sqrt{5})$ is Galois.
			
			Any automorphism in $\text{Aut}(K/\Q)$ must fix $\Q$. This excludes any function sending $\pm \sqrt{a}$ to $\pm \sqrt{b}$ when $a \not = b$. To see this, let $\phi$ be a function where $\phi(\sqrt{2}) = \sqrt{3}$. Then, $\phi(2) = \phi(\sqrt{2}\sqrt{2}) = 3$, meaning $\phi$ does not fix $\Q$.
			
			The remaining possible set of non trivial automorphisms are those swapping the signs of any root. Let such automorphism be defined as $\varphi$, $\sigma$, and $\tau$ swapping the signs of $\sqrt{2}$, $\sqrt{3}$, and $\sqrt{5}$ respectively, and $1$ being the identity. These automorphisms fix $\Q$ since $\phi\sigma\tau(a^2) = (-a)^2 = a^2$ for $a=2,3,5$.
			
			The Galois group is therefore all combinations of these functions, namely the set $\{1, \varphi, \sigma, \tau\, \varphi\sigma, \varphi\tau, \sigma\tau, \varphi\sigma\tau\}$. The subgroups of this are those generated by $\{\varphi\}, \{\sigma\}, \{\tau\}, \{\varphi,\sigma\}, \{\varphi,\tau\}, \{\sigma,\tau\}, \{\varphi\sigma\}, \{\varphi\tau\},\{\sigma\tau\},\{\tau, \varphi\sigma\},$
			$\{\sigma,\varphi\tau\}, \{\varphi, \sigma\tau\},$ and $\{\varphi\sigma\tau\}$.
			
			By the FTGT, there is a one to one correspondence between these subgroups and the subfields of $\Q(\sqrt{2},\sqrt{3},\sqrt{5})$, given by the fixed field of the subgroup. The first six fixed fields are easily seen to be $\Q(\sqrt{3},\sqrt{5}), \Q(\sqrt{2},\sqrt{5}), \Q(\sqrt{2},\sqrt{3}), \Q(\sqrt{5}), \Q(\sqrt{3}),$ and $\Q(\sqrt{2})$. The next six are given by considering the products of roots. For example, $\varphi\sigma(\sqrt{6}) = \varphi\sigma(\sqrt{2}\sqrt{3}) = (-\sqrt{2})(-\sqrt{3}) = \sqrt{6}$. All together, we have $\Q(\sqrt{5}, \sqrt{6}), \Q(\sqrt{3},\sqrt{10}),\Q(\sqrt{2},\sqrt{15}), \Q(\sqrt{6}), \Q(\sqrt{10}),\Q(\sqrt{15})$. The final subfield is given by $\Q(\sqrt{6},\sqrt{10},\sqrt{15})$
		\end{proof}
	\end{problem}
		
	\begin{problem}{2.5}
		Prove that the Galois group of $x^p-2$ for $p$ a prime is isomorphic to the group of matrices 
		$\begin{pmatrix}
			a & b\\
			0 & 1
		\end{pmatrix}$ where $a,b \in \mathbb{F}_p, a \not= 0$.
		\begin{proof}
			The splitting field of this polynomial is $\Q(\sqrt[p]{2}, \zeta_p)$, where $\sqrt[p]{2}$ is any fixed $p$th root of $2$ and $\zeta_p$ is the primitive $p$th root of unity. 
			
			From section 13.6, we know that the dimension of $\Q(\zeta_p)$ is $p-1$. It is also easy to see that $[\Q(\sqrt[p]{2}, \zeta_p):\Q(\zeta_p)]$ is $p$. Taken together, we have  $[\Q(\zeta_p,\sqrt[p]{2}):\Q] = [\Q(\zeta_p,\sqrt[p]{2}):\Q(\sqrt[p]{2})][\Q(\sqrt[p]{2}):\Q] = p(p-1)$.
			
			Since the polynomial $x^p-2$ is separable, $\Q(\sqrt[p]{2}, \zeta_p)$ is Galois. Hence, $[\Q(\sqrt[p]{2}, \zeta_p):\Q] = p(p-1) = \text{Aut}(\Q(\sqrt[p]{2}, \zeta_p)/\Q)$ There are hence $p(p-1)$ automorphism in $\text{Aut}(\Q(\sqrt[p]{2}, \zeta_p)/\Q)$.
			
			The Galois group is determined by the action on the generators $\sqrt[p]{2}$ and $\zeta_p$, lending possible automorphisms $\sigma_{a,b} : \zeta_p \mapsto \zeta_p^a, \sqrt[p]{2} \mapsto \zeta_p^b\sqrt[2]{p}$, where $0 < a < p$ and $0 \leq b < p$. (Letting $a$ equal $0$ would remove all primitive roots of unity from the field, so we can negate this option as not being an automorphism). We know the group is of order $p(p-1)$; hence, each $\sigma_{a,b}$ is distinct.
			
			Now, consider the function $\phi : \sigma_{a,b} \mapsto 	
			\begin{pmatrix}
				a & b\\
				0 & 1
			\end{pmatrix}$. We have constrained $a$ and $b$ in such a way that this function is obviously a bijection. So we need only show that it is an isomorphism. Note that $\sigma_{c,d}\sigma_{a,b}$ is the mapping $\zeta_p \mapsto \zeta_p^ca, \sqrt[p]{2} \mapsto \sigma_{a,b}(\zeta_p)^d\sigma_{a,b}(\sqrt[2]{p}) = \zeta^{ad+b}\sqrt[p]{2}$. So we can write it as $\sigma_{ac,bc+d}$ Now, for any $\sigma_{a,b}, \sigma_{c,d}$, we have $\phi(\sigma_{a,b})\phi(\sigma_{c,d}) = 
			\begin{pmatrix}
			a & b\\
			0 & 1
			\end{pmatrix}
			\begin{pmatrix}
			c & d\\
			0 & 1
			\end{pmatrix} = 
			\begin{pmatrix}
				ac & bc+d\\
				0 & 1
			\end{pmatrix} =
			\phi(\sigma_{ac, bc+1}) = \phi(\sigma_{c,d}\sigma_{a,b})$. Hence, the function is an isomorphism, completing the proof.
			
			\rem
			This proof took a while because I am not used to working with roots of unity; I understand they are very important in some areas of math. What is ironic, is that I barely did anything with the actual field, relying instead on the fundamental theorem of Galois theory.
		\end{proof} 
	\end{problem}
	
	\begin{problem}{2.7}
		Determine all the subfields of the splitting field of $x^8 - 2$ which are Galois.
		\begin{proof}
			From TFTGT, this is equivalent to finding the fixed fields of all normal subgroups of the Galois group of the splitting field for $x^8-2$. 
			
			We are given earlier in this chapter that the Galois group of this field is the quasihedral group defined by
			\begin{center}
				$\langle \sigma, \tau | \sigma^8 = \tau^2 = 1, \sigma\tau = \tau\sigma^3 \rangle$
			\end{center}
			//TODO
		\end{proof}
	\end{problem}
	
	\begin{problem}{2.9}
		Give an example of fields $\mathbb{F}_1, \mathbb{F}_2, \mathbb{F}_3$ with $\Q \subset \mathbb{F}_1 \subset \mathbb{F}_2 \subset \mathbb{F}_3$, $[\mathbb{F}_3 : \Q] = 8$ and each field is Galois over all its subfields with the exception that $\mathbb{F}_2$ is not Galois over $\Q$.
		\begin{proof}
			Consider $\mathbb{F}_3 = \Q(\sqrt[4]{2}, i), \mathbb{F}_2 = \Q(\sqrt[4]{2}), \mathbb{F}_1 = \Q(\sqrt{2})$. Clearly, this collection satisfies the chain of subset inclusions. The fields $\Q(\sqrt[4]{2})$ and $\Q(i)$ are degree 4 and 2 respectively. Since $i$ and $\sqrt[4]{2}$ are linearly independent, $[\mathbb{F}_3:\Q] = [\Q(\sqrt[4]{2}):\Q][\Q(i):\Q] = 4\cdot2 = 8$. $\mathbb{F}_3$ is the splitting field of $x^4-2$, $x^2 + \sqrt{2}$, and $x^4 - 1$ over $\Q$, $\mathbb{F}_1$, and $\mathbb{F}_2$ respectively. $\mathbb{F}_2$ is the splitting field of $x^2-\sqrt{2}$ over $\mathbb{F}_1$ is not a splitting field over $\Q$ since it does not contain $\pm i\sqrt[2]{2}$. Finally, $\mathbb{F}_1$ is the splitting field of $x^2-2$ over $\Q$. This completes the proof.
		\end{proof}
	\end{problem}
	
	\begin{problem}{2.11}
		Suppose $f(x) \in \Z[x]$ is an irreducible quartic whose splitting field has Galois group $S_4$ over $\Q$ (there are many such quartics, cf. Section 6). Let $\theta$ be a root of $f(x)$ and set $K = \Q(\theta)$. Prove that $K$ is an extension of $\Q$ of degree $4$ which has no proper subfields. Are there any Galois extensions of $\Q$ of degree $4$ with no proper subfields?
		\begin{proof}
			We write the polynomial in question as $(x - \theta)(x - \theta_1)(x - \theta_2)(x - \theta_3)$. The Galois subgroup associated with $K$ is the subset of $S_4$ fixing $\theta$, which is clearly $S_3$. If $K$ has a nontrivial subfield, then their is a nontrivial subgroup of $S_4$ containing $S_3$. Such a subgroup would be generated by $S_3$ and some function $\sigma$ swapping $\theta$ for another root. But this pair would generate $S_4$. Hence, no such subgroup exists and $K$ therefore has no proper subfields.
			
			To see that $K$ is degree $4$, note that by the fundamental theorem, $[K:\Q] = |S_4:S_3| = 4$
			
			If a Galois extension has degree $4$, then its Galois group would either be the cyclic four-group, or the Klein four-group, both of which have nontrivial subgroups. Thus, //TODO
		\end{proof}
	\end{problem}
	
	\begin{problem}{2.13}
		Prove that if the Galois group of the splitting field of a cubic over $\Q$ is the cyclic group of order $3$ then all the roots of the cubic are real.
		\begin{proof}
			Let $p(x)$ be a cubic polynomial in $\Q[x]$. Suppose that $p(x)$ has at least one imaginary root $r_1=a+bi$. From calculus, we know $p(x)$ must have at least one real root, call this root $r_2$. Label the remaining root as $r_3$. Then $p(x) = (x-r_1)(x-r_2)(x-(a+bi))$. It is easy to verify from this that $r_2 = a-bi$, the conjugate of $r_1$. But then, the automorphism defined by $\sigma: a+bi \mapsto a-bi$ must be an element of the Galois group. This automorphism fixes $r_1$, hence the Galois group is not isomorphic to the cyclic group of order $3$.
		\end{proof}
	\end{problem}
	
	\begin{problem}{2.15}
		(\textit{Biquadratic Extensions}) Let $F$ be a field of characteristic $\not = 2$.
		\begin{itemize}
			\item[\textbf{(a)}] If $K=F(\sqrt{D_1}, \sqrt{D_2})$ where $D_1, D_2 \in F$ have the property that none of $D_1, D_2,$ or $D_1d_2$ are square in $F$, prove that $K/F$ is a Galois extension with $\Gal(K/F)$ isomorphic to the Klein 4-group.
			\item[\textbf{(b)}] Conversely, suppose $K/F$ is a Galois extension with $\Gal(K/F)$ isomorphic to the Klein 4-group. Prove that $K = F(\sqrt{D_1}, \sqrt{D_2})$ where $D_1, D_2 \in F$ has the property that none of $D_1, D_2,$ or $D_1D_2$ is square in $F$.
		\end{itemize}
		\begin{proof}{(a)}
			The minimal polynomial of $F(\sqrt{D_1}, \sqrt{D_2})$ is the reducible expression $(x^2-D_1)(x^2-D_1)$. Hence $F(\sqrt{D_1})$ and $F(\sqrt{D_2})$ are Galois sub fields of degree $2$ over $F$ and their intersection is $F$. This means the Galois group of $K$ has two mutually exclusive normal subgroups of order two. These two subgroups are generated by namely, the automorphisms swapping the sign of the roots. Clearly, these automorphism combine to create the fourth element of the group, which must be its own inverse. Writing out the Caley table of this group yields
			\begin{center}
				\noindent\begin{tabular}{c | c c c c}
								& e          & $\sigma_1$ & $\sigma_2$ & $\sigma_3$  \\
					\cline{1-4}
					e   		& e          & $\sigma_1$ & $\sigma_2$ & $\sigma_3$  \\
					$\sigma_1$  & $\sigma_1$ & e 		  & $\sigma_3$ & $x$ \\
					$\sigma_2$  & $\sigma_2$ & $\sigma_3$ & e 		   & $x$ \\
					$\sigma_3$ 	& $\sigma_3$ & $x$        & $x$        & e \\
				\end{tabular}
			\end{center}
			From this, the last four values can be deduced yielding a table of
			\begin{center}
				\noindent\begin{tabular}{c | c c c c}
					& e          & $\sigma_1$ & $\sigma_2$ & $\sigma_3$  \\
					\cline{1-4}
					e   		& e          & $\sigma_1$ & $\sigma_2$ & $\sigma_3$  \\
					$\sigma_1$  & $\sigma_1$ & e 		  & $\sigma_3$ & $\sigma_2$ \\
					$\sigma_2$  & $\sigma_2$ & $\sigma_3$ & e 		   & $\sigma_1$ \\
					$\sigma_3$ 	& $\sigma_3$ & $\sigma_1$ & $\sigma_2$ & e \\
				\end{tabular}
			\end{center} 
		\end{proof}
		\begin{proof}[(b)]
			The field $K$ is Galois, so we can label the roots of its minimal polynomial as $r_1, r_2, r_3,$ and $r_4$. Since every element of the Klein $4$-group is its own inverse, every automorphism in $\Gal(K/F)$ must act as a collection of mutually exclusive $2$-cycles on the roots. Since any two different elements generate the third element, it is clear that abusing notation, the automorphisms take the form $\sigma_1 = (r_1r_2), \sigma_2 = (r_3r_4),$ and $\sigma_3 = (r_1r_2)(r_3r_4)$. 
			
			These three elements generate normal subgroups and hence are each associated with Galois fields. The first two fields are $F(r_3, r_4)$ and $F(r_1, r_2),$. Since each subfield is of dimension $2$ over $F$, the minimal polynomials of each are $(x-r_3)(x-r_4)$ and $(x-r_1)(x-r_2),$. From this, it is clear that $r_1 = -r_2 = \sqrt{D_1}, r_3=-r_4=\sqrt{D_2},$. Finally, we can see that the third subfield is $F(\sqrt{D_1D_2})$, which is of degree $2$. Hence $D_1D_2$ is not square in $F$.	
		\end{proof}
	\end{problem}
	
	\begin{problem}{2.17}
		Let $K/F$ be any finite extension and let $a \in K$. Let $L$ be a Galois extension of $F$ containing $K$ and let $H \leq \Gal(L/F)$ be the subgroup corresponding to $K$ . Define the \textit{norm} of $a$ from K to F to be
		\begin{align*}
			N_{K/F}(a) = \prod_{\sigma} \sigma(a),
		\end{align*}
		where the product is taken over all the embeddings of $K$ into an algebraic closure of $F$ (so over a set of coset representatives for $H$ in $\Gal(L/F)$ by the Fundamental Theorem	of Galois Theory). This is a product of Galois conjugates of $a$. In particular, if $K/F$ is	Galois this is $\prod_{\sigma \in \Gal(K/F)} \sigma(a)$.
		\begin{itemize}
			\item[\textbf{(a)}] Prove that $N_{k/f}(\alpha) \in F$.
			\item[\textbf{(b)}] Prove that $N_{k/F}(\alpha \beta) = N_{K/F}(\alpha)N_{K/F}(\beta)$, so that the norm is a multiplicative map from $K$ to $F$.
			\item[\textbf{(c)}] Let $K = F(\sqrt{D})$ be a quadratic extension of $F$. Show that $N_{K/F}(a+b\sqrt{D}) = a^2 - b^2D$.
			\item[\textbf{(d)}] Let $m_{\alpha}(x) = x^d + a_{d-1}x^{d-1} + ... + a_1x + a_0 \in F[x]$ be the minimal polynomial for $\alpha \in K$ over $F$. Let $n=[K:F]$ Prove that $d$ divides $n$, that there are $d$ distinct Galois conjugates of $\alpha$ which are repeated $n/d$ times in the product above and conclude that $N_{K/F}(\alpha) = (-1)^na_0^{n/d}.$
		\end{itemize}
		\begin{proof}{(a)} If the element $\alpha$ is in $F$, then the proof is trivial. Otherwise, consider the minimal polynomial of $\alpha$, $p(x) = (x-a_1)(x-a_2)...(x-a_n)(x-\alpha)$. Since this polynomial is in $F[x]$, the coefficients and therefore the element $a_1a_2...a_n\alpha$ is in $F$. But this element is precisely the product $\prod_{\sigma} \sigma(a) = N_{k/f}(\alpha)$. Hence, the norm of $\alpha$ is in $F$.
		\end{proof}
		\begin{proof}[(b)]
			Since each $\sigma$ is an automorphism, we have $N_{K/F}(\alpha \beta) = \prod_{\sigma} \sigma(\alpha \beta) = \prod_{\sigma} \sigma(\alpha) \sigma (\beta) = \prod_{\sigma} \sigma(\alpha) \prod_{\sigma} \sigma (\beta) = N_{K/F}(\alpha) N_{K/F}(\beta)$. 
		\end{proof}
		\begin{proof}[(c)]
			We have that $K$ is the splitting field for the minimal polynomial $x^2 - D = (x-\sqrt{D})(x+\sqrt{D})$, and $K$ is Galois. The Galois group of $K/F$ consists of the automorphism switching the sign of $\sqrt{D}$ and the identity. Label these as $\sigma$ and $\tau$ respectively. Then, $\N_{K/F}(a+b\sqrt{D}) = \sigma(a+b\sqrt{D})\tau(a+b\sqrt{D}) = (a-b\sqrt{D})(a+b\sqrt{D}) = a^2 - b^2D$.
		\end{proof}
		\begin{proof}[(d)]
			The proof is trivial for $\alpha \in F$ since $d$ would be $0$. Assume then that $\alpha \not \in F$. The splitting field of $m_\alpha(x)$ is then $F(\alpha)$, which is obviously a subfield of $K$. Hence $n = [F:K] = [F:F(\alpha)][E:K] = d[F(\alpha):K]$. This gives us $n/d = [F(\alpha):K]$, hence $d$ divides $n$.
			
			The minimal polynomial is irreducible and separable, so there are $d$ distinct roots not in $F$. Hence, there are $d$ distinct conjugates of $\alpha$.
			
			Let $G \leq \Gal(L/F)$ be the group corresponding to $F(\alpha)$. In the above product, the group $\Gal(L/F)/G$ forms the set of unique permutations of $\alpha$. The size of the cosets in this group is the number of times a distinct conjugation is repeated. This is the same as $[\Gal(L/F)/H:\Gal(L/F)/G] = |G:H|$ We have that $n = [K:F] = |\Gal(L/F):H|$ and $d = [F(\alpha):F] = |\Gal(L/F):G|$. Hence, $|G:H| = n/d$.
			
		\end{proof}
		
		\rem
			I had trouble figuring out what was meant by "the product is taken over all the embeddings of $K$ into an algebraic closure of $F$." I'm still not sure if I interpreted it right, but from context clues, I believe that the $\sigma$ are the elements of $\Gal(L/F)/H$.
	\end{problem}
	
	\begin{problem}{2.19}
		With notation as in the previous problems show that $N_{K/F}(a\alpha) = a^nN_{K/F} (\alpha)$ and $Tr_{K/F}(a\alpha) = aTr_{K/F}(\alpha)$ for all $a$ in the base field $F$, In particular show that $N_K/F(a) = a^n$ and $Tr_{K/F}(a) = na$ for all $a \in F$.
		\begin{proof}
			We are given that $n = [K:F] = |\Gal(L/F):H|$ hence, $|\Gal(L/F)/H| = n$. So there are $n$ terms in the product and sum of the norm and trace resp. We can thus rewrite both as
			\begin{align*}
				N_{K/F}(a\alpha) = \sigma_1(a\alpha)...\sigma_n(a\alpha) = a^n\sigma_1(\alpha)...\sigma_n(\alpha) = a^nN_{K/F}(\alpha)
			\end{align*}
			and
			\begin{align*}
				Tr_{K/F}(a\alpha) = \sigma_1(a\alpha) + ... + \sigma_n(a\alpha) = a(\sigma(\alpha) + ... + \sigma_n(\alpha)) = aTr_{K/F}(\alpha)
			\end{align*}
			The rest of the proof is trivial for $a \in F$.
		\end{proof}
	\end{problem}
	
	\begin{problem}{2.21}
		Use the linear independence of characters to show that for any Galois extension $K$ of $F$ there is an element $\alpha \in K$ with $Tr_{K/F}(\alpha) \not= 0$.
		\begin{proof}
			By definition, the trace is a sum of embeddings $\sigma_1(\alpha) + ... + \sigma_n(\alpha)$. By corollary 8, these embeddings must be linearly independent, hence, there is a nonzero $\alpha \in K$ for which $Tr_{K/F}(\alpha) \not = 0$.
		\end{proof}
	\end{problem}
	
	\begin{problem}{2.23}
		(Hilbert's Theorem 90) Let $K$ be a Galois extension of $F$ with cyclic Galois group of order $n$ generated by $\sigma$. Suppose $\alpha \in K$ has $N_{K/F}(a)=1$. Prove that $\alpha$ is of the form $\frac{\beta}{\sigma\beta}$ for some nonzero $\beta \in K$. [By the linear independence of characters show there exists some $\theta \in K$ such that 
		\begin{align*}
			\beta = \theta + \alpha\sigma(\theta) + (\alpha\sigma\alpha)\sigma^2(\theta) + ... + (\alpha\sigma\alpha...\sigma^{n-2}\alpha)\sigma^{n-1}(\theta)
		\end{align*}
		is nonzero. Compute $\frac{\beta}{\sigma\beta}$ using the fact that $\alpha$ has norm $1$.
		\begin{proof}
			Since $\sigma$ generates the Galois group and it is of order $n$, the set $\sigma^0, \sigma^1, ..., \sigma^{n-1}$ represents distinct characters and are therefore linearly independent. Hence, for the function
			\begin{align*}
				\varphi: x \mapsto x + \alpha\sigma(x) + (\alpha\sigma\alpha)\sigma^2(x) + ... + (\alpha\sigma\alpha...\sigma^{n-2}\alpha)\sigma^{n-1}(x),
			\end{align*}
			there is some $\theta$ for which $\varphi(\theta) \not= 0$. Denote $\varphi(\theta)$ as $\beta$. Taking $\sigma$ of $\beta$ yields
			\begin{align*}
				\sigma(\theta + \alpha\sigma(\theta) + (\alpha\sigma\alpha)\sigma^2(\theta) + ... + (\alpha\sigma\alpha...\sigma^{n-2}\alpha)\sigma^{n-1}(\theta)) = 
			\end{align*}
			\begin{align*}
				\sigma(\theta) + (\sigma\alpha)\sigma^2(\theta) + (\sigma\alpha\sigma^2\alpha)\sigma^3(\theta) + ... + (\sigma\alpha\sigma^2\alpha...\sigma^{n-1}\alpha)\sigma^{n}(\theta)
			\end{align*}
			The last term in this sum is equivalent to $N_{K/F}(\alpha)\sigma^n(\theta)$. Since the norm of $\alpha$ is given to be $1$ and $\sigma^n$ is the identity, this term becomes merely $\theta$. Making this substitution and reordering the terms yields
			\begin{align*}
				\theta + \sigma(\theta) + (\sigma\alpha)\sigma^2(\theta) + (\sigma\alpha\sigma^2\alpha)\sigma^3(\theta) + ... + (\sigma\alpha\sigma^2\alpha...\sigma^{n-2}\alpha)\sigma^{n-1}(\theta),
			\end{align*}
			which is clearly $\beta/\alpha$. Hence, $\sigma(\beta) = \beta/\alpha$ and therefore $\frac{\beta}{\sigma\beta} = \alpha$.
		\end{proof}
	\end{problem}
\end{document}
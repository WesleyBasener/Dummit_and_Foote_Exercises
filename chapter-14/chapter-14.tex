\documentclass[10pt]{article}

\usepackage[margin=1in]{geometry} 
\usepackage{amsmath,amsthm,amssymb, graphicx, multicol, array}
\usepackage{tikz-cd} 

\newcommand{\N}{\mathbb{N}}
\newcommand{\Z}{\mathbb{Z}}
\newcommand{\Q}{\mathbb{Q}}
\newcommand{\R}{\mathbb{R}}
\newcommand{\C}{\mathbb{C}}

\newenvironment{problem}[2][Problem]{\begin{trivlist}
		\item[\hskip \labelsep {\bfseries #1}\hskip \labelsep {\bfseries #2.}]}{\end{trivlist}}

\newenvironment{answer}[2][Answer]{\begin{trivlist}
		\item[\hskip \labelsep {\bfseries #1}\hskip \labelsep {\bfseries #2.}]}{\end{trivlist}}

\newenvironment{theorem}[2][Theorem]{\begin{trivlist}
		\item[\hskip \labelsep {\bfseries #1}\hskip \labelsep {\bfseries #2.}]}{\end{trivlist}}

\newtheorem{thm}{Theorem}
\newtheorem{defn}{Definition}
\newtheorem{conv}{Convention}
\newtheorem{rem}{Remark}
\newtheorem{lem}{Lemma}
\newtheorem{cor}{Corollary}

\begin{document}
	
	\title{Exercises from Dummit and Foote Chapter 14 on Galois Theory}
	\author{Wesley Basener}
	\maketitle
	
	\begin{problem}{1.1}
		\begin{itemize}
			\item[\textbf{(a)}]
				Show that if the field $K$ is generated over $F$ by the elements $\alpha_1, ..., \alpha_n$ then an
				automorphism $\sigma$ of $K$ fixing $F$ is uniquely determined by $\sigma (\alpha_1), ..., \sigma (\alpha_n )$. In
				particular show that an automorphism fixes $K$ if and only if it fixes a set of generators
				for $K$.
			\item[\textbf{(b)}]
				Let $G \leq \text{Gal}(K/F)$ b e a subgroup of the Galois group of the extension $K/F$ and
				suppose $\sigma_1, ..., \sigma_k$ are generators for $G$. Show that the subfield $E/F$ is fixed by $G$ if
				and only if it is fixed by the generators $\sigma_1, ..., \sigma_k$.
		\end{itemize}
		\begin{proof}{(a)}
			  Let $\sigma$ be any automorphism on $K$ fixing $F$. Then, for any $k = a_0 + a_1\alpha_1 + ... + a_n\alpha_n$ in $K$, $\sigma(k) = \sigma(a_0) + \sigma(a_1)\sigma(\alpha_1) + ... + \sigma(a_n)\sigma(\alpha_n)$. Using the fact that $\sigma$ fixes $F$, we have $\sigma(k) = a_0 + a_1\sigma(\alpha_1) + ... + a_n\sigma(\alpha_n)$. Hence the image of any $k \in K$ on $\sigma$ is uniquely determined by $\sigma (\alpha_1), ..., \sigma (\alpha_n )$.
			  From this, it is obvious that $\sigma$ fixes $K$ if it fixes the generators for $K$.
		\end{proof}
		\begin{proof}[(b)]
			Denote the generators of $E$ over $F$ by $\alpha_1, ..., \alpha_m$. Suppose $G$ fixes $E/F$. From part (a), this is true if and only if $\sigma_i(\alpha_j) = \alpha_j$ for all $i \in [1,k], j \in [1,m]$. Hence, any element of a $E/F$ is fixed by any element of $G$.
		\end{proof}
	\end{problem}
	
	\begin{problem}{1.3}
		Determine the fixed field of complex conjugation on $\C$.
		\begin{proof}
			Complex conjugation is the function $\sigma : a+bi \mapsto a-bi$, which obviously fixes $a$. Hence, the fixed field of complex conjugation is $\R$ the real numbers.
		\end{proof}
	\end{problem}
	
	\begin{problem}{1.5}
		Determine the automorphisms of the extension $\Q(\sqrt[4]{2})/\Q(\sqrt{2})$ explicitly.
		\begin{proof}
			There is only one basis element to this extension, namely $\sqrt[4]{2}$. Since $-\sqrt[4]{2} \not = \sqrt[4]{2}$, the automorphism $\sigma : a + b\sqrt[4]{2} \mapsto a - b\sqrt[4]{2}$ is not the identity. Hence, the automorphisms of this extension are $\{1, \sigma\}$.
		\end{proof}
	\end{problem}
	
	\begin{problem}{1.7}
		This problem determines $\text{Aut}(\R/\Q)$.
		\begin{itemize}
			\item[\textbf{(a)}]
				Prove that any $\sigma \in \text{Aut}(\R/\Q)$ takes squares to squares and takes positive reals to positive reals. Conclude that $a < b$ implies $\sigma a < \sigma b$ for every $a, b \in \R$ Conclude that $a<b$ implies $\sigma a<\sigma b$ for every $a,b \in \R$.
			\item[\textbf{(b)}]
				Prove that $-\frac{1}{m} < a - b < \frac{1}{m}$ implies $-\frac{1}{m} < \sigma a - \sigma b < \frac{1}{m}$ for every positive integer $m$. Conclude that $\sigma$ is a continuous map on $\R$.
			\item[\textbf{(c)}]
				Prove that any continuous map on $\R$ which is the identity on $\Q$ is the identity map, hence $\text{Aut}(\R/\Q)=1$. 
		\end{itemize}
		\begin{proof}{(a)}
			Let $\sigma$ be an automorphism on $\R/\Q$. Suppose $x$ is a real square. Then, $x = p^2$ for real number $p$. Hence, we have $\sigma(x) = \sigma(p^2) = \sigma(p)\sigma(p)$. Thus, $\sigma$ sends squares to squares. 
			
			Let $y$ be any positive real number. Since $y$ is positive, $\sqrt{y}$ is real. From the first part of this proof, we know that $\sigma(y) = \sigma(\sqrt{y}\sqrt{y}) = q^2$ for some real number $q$. Since we are limited to the real numbers, $q^2$ is positive. Hence, $\sigma(y)$ is positive.
			
			For any $a,b \in \R$, $a<b$ implies $0 < b-a$. Hence, from the prior paragraph, $0 < \sigma(b) - \sigma(a)$. Adding $\sigma(a)$ to both sides yields $\sigma(a) < \sigma(b)$. Note that setting $b=0$ proves that $\sigma$ sends negatives to negatives.
		\end{proof}
		\begin{proof}[(b)]
			Suppose $a,b$ are real numbers such that $-\frac{1}{m} < a - b < \frac{1}{m}$ for some positive integer $m$.
			
			Since $m$ is an integer, it can be rewritten as $\Sigma_{i=0}^m 1$. Hence, $\sigma(m) = \Sigma_{i=0}^m \sigma(1) = \Sigma_{i=0}^m 1 = m$.
			
			We can rewrite the above inequality as $-1 < m(a-b) < 1$. Which is the same as having $m(a-b) - 1$ is negative and $m(a-b) + 1$ is positive. From part (a), we know $\sigma$ sends positives to positives and negatives to negatives. Hence, $\sigma(m(a-b) - 1) = m(\sigma(a) - \sigma(b)) - 1$ is negative and $\sigma(m(a-b) - 1) = m(\sigma(a) - \sigma(b)) + 1$ is positive. Which of course implies $-\frac{1}{m} < \sigma(a) - \sigma(b) < \frac{1}{m}$
			
			To show that $\sigma$ is continuous, let $x$ be any real number and let $\epsilon > 0$. We can find a natural number $N$ such that $\frac{1}{N} < \epsilon$. Then, for any $x_0$ such that $|x-x_0| < \frac{1}{N}$, we have $|\sigma(x) - \sigma(x_0)| < \frac{1}{N} < \epsilon$. Hence, $\sigma$ is continuous.
		\end{proof}
		\begin{proof}[(c)]
			Suppose $\sigma \in \text{Aut}(\R/\Q)$ fixes $\Q$. Let $x$ be any real number. Then by the density of the rationals in $\R$, for any $\epsilon > 0$, there exists some $q \in \Q$ such that $|x-q| < \epsilon$ Hence, $|\sigma(x-q)| = |\sigma(x)-q| < \epsilon$ which is only possible if $\sigma(x) = x$. Thus, any such $\sigma$ mus be the identity function. Therefore, $\text{Aut}(\R/\Q)=1$.
		\end{proof}
	\end{problem}
	
	\begin{problem}{1.9}
		Determine the fixed field of the automorphism $t \mapsto t + 1$ of $k(t)$.
		\begin{proof}
			Any element of $k(t)$ will have the form $\frac{\Sigma a_i t^i}{\Sigma b_i t^i}$ with $\gcd(\Sigma a_i x^i, \Sigma b_i x^i) = 1$. 
			Suppose we have an element such that $\frac{\Sigma a_i (t+1)^i}{\Sigma b_i (t+1)^i} = \frac{\Sigma a_i t^i}{\Sigma b_i t^i}$. Then, $\frac{\Sigma a_i (t+1)^i}{\Sigma b_i (t+1)^i} - \frac{\Sigma a_i t^i}{\Sigma b_i t^i} = 0$ and since both fractions remain irreducible, we would have $\Sigma b_i (t+1)^i = \Sigma b_i t^i$. Thus, we would also have $\Sigma a_i (t+1)^i = \Sigma a_i (t)^i$. Hence, the fixed field of $k(t)$ is precisely the set of rational functions whose numerators and denominators are both fixed by the automorphism.
			
			//TODO: finish this proof.
		\end{proof}
	\end{problem}
	
	\begin{problem}{2.1}
		Determine the minimal polynomial over $\Q$ for the element .
		\begin{proof}
			We have that $\Q(\sqrt{2}+\sqrt{5})$ is a subfield of $\Q(\sqrt{2}, \sqrt{5})$, which is the splitting field of $(x^2-2)(x^2-5)$. Since this polynomial is separable, $\Q(\sqrt{2}, \sqrt{5})$ is Galois.
			
			We can therefore find the other roots of the minimal polynomial of $\Q(\sqrt{2}+\sqrt{5})$ by considering the action of $\text{Aut}(\Q/\Q(\sqrt{2}, \sqrt{5})$ on $\sqrt{2}+\sqrt{5}$. This yields $\pm\sqrt{2} \pm\sqrt{5}$, which are indeed distinct. 
			
			Hence, the minimal polynomial of $\sqrt{2}+\sqrt{5}$ is $(x - \sqrt{2}+\sqrt{5})(x + \sqrt{2}+\sqrt{5})(x - \sqrt{2}-\sqrt{5})(x + \sqrt{2}-\sqrt{5})$ which multiplies to $x^4 - 14x^2 + 9$.
			
			
			\rem
				The inverse of $\sqrt{2}+\sqrt{5}$ on $\Q(\sqrt{2}+\sqrt{5})$ is $\frac{\sqrt{2}-\sqrt{5}}{-3}$. Hence, the field $\Q(\sqrt{2}+\sqrt{5})$ contains $\sqrt{5}$ and $\sqrt{2}$. Given that $\Q(\sqrt{2}+\sqrt{5})$ is a subfield of $\Q(\sqrt{2}, \sqrt{5})$, we have that $\Q(\sqrt{2} + \sqrt{5}) = \Q(\sqrt{2}, \sqrt{5})$.
			
				From this, I initially though that the minimal polynomial of $\sqrt{2}+\sqrt{5}$ would be the same as the minimal polynomial with roots $\sqrt{2}$ and $\sqrt{5}$. But this is obviously not the case since $(\sqrt{2}+\sqrt{5})$ is not a root of $(x^2 - 5)(x^2 - 2)$.
				
				This is a case of being disillusioned of unjustified assumptions. Just because $F(a) = F(b,c)$, does not mean that the minimal polynomial of $a$ and the minimal polynomial with roots $b,c$ are the same. In this case, $(x^2 - 5)(x^2 - 2)$ is not reducible, so it is not a minimal polynomial for anything.
		\end{proof}
	\end{problem}
	
	\begin{problem}{2.3}
		Determine the Galois group of $(x^2 - 2)(x^2 - 3)(x^2 - 5)$. Determine all the subfields of the splitting field of this polynomial.
		\begin{proof}
			This polynomial is separable with roots $\pm\sqrt{2}$, $\pm\sqrt{3}$, and $\pm\sqrt{5}$. Hence, its splitting field $K = \Q(\sqrt{2},\sqrt{3},\sqrt{5})$ is Galois.
			
			Any automorphism in $\text{Aut}(K/\Q)$ must fix $\Q$. This excludes any function sending $\pm \sqrt{a}$ to $\pm \sqrt{b}$ when $a \not = b$. To see this, let $\phi$ be a function where $\phi(\sqrt{2}) = \sqrt{3}$. Then, $\phi(2) = \phi(\sqrt{2}\sqrt{2}) = 3$, meaning $\phi$ does not fix $\Q$.
			
			The remaining possible set of non trivial automorphisms are those swapping the signs of any root. Let such automorphism be defined as $\varphi$, $\sigma$, and $\tau$ swapping the signs of $\sqrt{2}$, $\sqrt{3}$, and $\sqrt{5}$ respectively, and $1$ being the identity. These automorphisms fix $\Q$ since $\phi\sigma\tau(a^2) = (-a)^2 = a^2$ for $a=2,3,5$.
			
			The Galois group is therefore all combinations of these functions, namely the set $\{1, \varphi, \sigma, \tau\, \varphi\sigma, \varphi\tau, \sigma\tau, \varphi\sigma\tau\}$. The subgroups of this are those generated by $\{\varphi\}, \{\sigma\}, \{\tau\}, \{\varphi,\sigma\}, \{\varphi,\tau\}, \{\sigma,\tau\}, \{\varphi\sigma\}, \{\varphi\tau\},\{\sigma\tau\},\{\tau, \varphi\sigma\},$
			$\{\sigma,\varphi\tau\}, \{\varphi, \sigma\tau\},$ and $\{\varphi\sigma\tau\}$.
			
			By the FTGT, there is a one to one correspondence between these subgroups and the subfields of $\Q(\sqrt{2},\sqrt{3},\sqrt{5})$, given by the fixed field of the subgroup. The first six fixed fields are easily seen to be $\Q(\sqrt{3},\sqrt{5}), \Q(\sqrt{2},\sqrt{5}), \Q(\sqrt{2},\sqrt{3}), \Q(\sqrt{5}), \Q(\sqrt{3}),$ and $\Q(\sqrt{2})$. The next six are given by considering the products of roots. For example, $\varphi\sigma(\sqrt{6}) = \varphi\sigma(\sqrt{2}\sqrt{3}) = (-\sqrt{2})(-\sqrt{3}) = \sqrt{6}$. All together, we have $\Q(\sqrt{5}, \sqrt{6}), \Q(\sqrt{3},\sqrt{10}),\Q(\sqrt{2},\sqrt{15}), \Q(\sqrt{6}), \Q(\sqrt{10}),\Q(\sqrt{15})$. The final subfield is given by $\Q(\sqrt{6},\sqrt{10},\sqrt{15})$
		\end{proof}
	\end{problem}
		
	\begin{problem}{2.5}
		Prove that the Galois group of $x^p-2$ for $p$ a prime is isomorphic to the group of matrices 
		$\begin{pmatrix}
			a & b\\
			0 & 1
		\end{pmatrix}$ where $a,b \in \mathbb{F}_p, a \not= 0$.
		\begin{proof}
			The splitting field of this polynomial is $\Q(\sqrt[p]{2}, \zeta_p)$, where $\sqrt[p]{2}$ is any fixed $p$th root of $2$ and $\zeta_p$ is the primitive $p$th root of unity. 
			
			From section 13.6, we know that the dimension of $\Q(\zeta_p)$ is $p-1$. It is also easy to see that $[\Q(\sqrt[p]{2}, \zeta_p):\Q(\zeta_p)]$ is $p$. Taken together, we have  $[\Q(\zeta_p,\sqrt[p]{2}):\Q] = [\Q(\zeta_p,\sqrt[p]{2}):\Q(\sqrt[p]{2})][\Q(\sqrt[p]{2}):\Q] = p(p-1)$.
			
			Since the polynomial $x^p-2$ is separable, $\Q(\sqrt[p]{2}, \zeta_p)$ is Galois. Hence, $[\Q(\sqrt[p]{2}, \zeta_p):\Q] = p(p-1) = \text{Aut}(\Q(\sqrt[p]{2}, \zeta_p)/\Q)$ There are hence $p(p-1)$ automorphism in $\text{Aut}(\Q(\sqrt[p]{2}, \zeta_p)/\Q)$.
			
			The Galois group is determined by the action on the generators $\sqrt[p]{2}$ and $\zeta_p$, lending possible automorphisms $\sigma_{a,b} : \zeta_p \mapsto \zeta_p^a, \sqrt[p]{2} \mapsto \zeta_p^b\sqrt[2]{p}$, where $0 < a < p$ and $0 \leq b < p$. (Letting $a$ equal $0$ would remove all primitive roots of unity from the field, so we can negate this option as not being an automorphism). We know the group is of order $p(p-1)$; hence, each $\sigma_{a,b}$ is distinct.
			
			Now, consider the function $\phi : \sigma_{a,b} \mapsto 	
			\begin{pmatrix}
				a & b\\
				0 & 1
			\end{pmatrix}$. We have constrained $a$ and $b$ in such a way that this function is obviously a bijection. So we need only show that it is an isomorphism. Note that $\sigma_{c,d}\sigma_{a,b}$ is the mapping $\zeta_p \mapsto \zeta_p^ca, \sqrt[p]{2} \mapsto \sigma_{a,b}(\zeta_p)^d\sigma_{a,b}(\sqrt[2]{p}) = \zeta^{ad+b}\sqrt[p]{2}$. So we can write it as $\sigma_{ac,bc+d}$ Now, for any $\sigma_{a,b}, \sigma_{c,d}$, we have $\phi(\sigma_{a,b})\phi(\sigma_{c,d}) = 
			\begin{pmatrix}
			a & b\\
			0 & 1
			\end{pmatrix}
			\begin{pmatrix}
			c & d\\
			0 & 1
			\end{pmatrix} = 
			\begin{pmatrix}
				ac & bc+d\\
				0 & 1
			\end{pmatrix} =
			\phi(\sigma_{ac, bc+1}) = \phi(\sigma_{c,d}\sigma_{a,b})$. Hence, the function is an isomorphism, completing the proof.
			
			\rem
			This proof took a while because I am not used to working with roots of unity; I understand they are very important in some areas of math. What is ironic, is that I barely did anything with the actual field, relying instead on the fundamental theorem of Galois theory.
		\end{proof} 
	\end{problem}
	
	\begin{problem}{2.7}
		Determine all the subfields of the splitting field of $x^8 - 2$ which are Galois.
		\begin{proof}
			From TFTGT, this is equivalent to finding the fixed fields of all normal subgroups of the Galois group of the splitting field for $x^8-2$. 
			
			We are given earlier in this chapter that the Galois group of this field is the quasihedral group defined by
			\begin{center}
				$\langle \sigma, \tau | \sigma^8 = \tau^2 = 1, \sigma\tau = \tau\sigma^3 \rangle$
			\end{center}
			//TODO
		\end{proof}
	\end{problem}
	
	\begin{problem}{2.9}
		Give an example of fields $\mathbb{F}_1, \mathbb{F}_2, \mathbb{F}_3$ with $\Q \subset \mathbb{F}_1 \subset \mathbb{F}_2 \subset \mathbb{F}_3$, $[\mathbb{F}_3 : \Q] = 8$ and each field is Galois over all its subfields with the exception that $\mathbb{F}_2$ is not Galois over $\Q$.
		\begin{proof}
			Consider $\mathbb{F}_3 = \Q(\sqrt[4]{2}, i), \mathbb{F}_2 = \Q(\sqrt[4]{2}), \mathbb{F}_1 = \Q(\sqrt{2})$. Clearly, this collection satisfies the chain of subset inclusions. The fields $\Q(\sqrt[4]{2})$ and $\Q(i)$ are degree 4 and 2 respectively. Since $i$ and $\sqrt[4]{2}$ are linearly independent, $[\mathbb{F}_3:\Q] = [\Q(\sqrt[4]{2}):\Q][\Q(i):\Q] = 4\cdot2 = 8$. $\mathbb{F}_3$ is the splitting field of $x^4-2$, $x^2 + \sqrt{2}$, and $x^4 - 1$ over $\Q$, $\mathbb{F}_1$, and $\mathbb{F}_2$ respectively. $\mathbb{F}_2$ is the splitting field of $x^2-\sqrt{2}$ over $\mathbb{F}_1$ is not a splitting field over $\Q$ since it does not contain $\pm i\sqrt[2]{2}$. Finally, $\mathbb{F}_1$ is the splitting field of $x^2-2$ over $\Q$. This completes the proof.
		\end{proof}
	\end{problem}
	
	\begin{problem}{2.11}
		Suppose $f(x) \in \Z[x]$ is an irreducible quartic whose splitting field has Galois group $S_4$ over $\Q$ (there are many such quartics, cf. Section 6). Let $\theta$ be a root of $f(x)$ and set $K = \Q(\theta)$. Prove that $K$ is an extension of $\Q$ of degree $4$ which has no proper subfields. Are there any Galois extensions of $\Q$ of degree $4$ with no proper subfields?
		\begin{proof}
			We write the polynomial in question as $(x - \theta)(x - \theta_1)(x - \theta_2)(x - \theta_3)$. The Galois subgroup associated with $K$ is the subset of $S_4$ fixing $\theta$, which is clearly $S_3$. If $K$ has a nontrivial subfield, then their is a nontrivial subgroup of $S_4$ containing $S_3$. Such a subgroup would be generated by $S_3$ and some function $\sigma$ swapping $\theta$ for another root. But this pair would generate $S_4$. Hence, no such subgroup exists and $K$ therefore has no proper subfields.
			
			To see that $K$ is degree $4$, note that by the fundamental theorem, $[K:\Q] = |S_4:S_3| = 4$
			
			If a Galois extension has degree $4$, then its Galois group would either be the cyclic four-group, or the Klein four-group, both of which have nontrivial subgroups. Thus, //TODO
		\end{proof}
	\end{problem}
	
	\begin{problem}{2.13}
		Prove that if the Galois group of the splitting field of a cubic over $\Q$ is the cyclic group of order $3$ then all the roots of the cubic are real.
		\begin{proof}
			Let $p(x)$ be the polynomial in question. Suppose for the sake of contradiction that $p(x)$ has at least one imaginary root. From calculus, we know $p(x)$ must have at least one real root. By assumption, there is an automorphism $\sigma$ sending an imaginary root to the real one. But such a function cannot be an automorphism because $\sigma(i)^2 = -1$, where $\sigma(i)$ is real. Hence, all roots of $p(x)$ are real.
		\end{proof}
	\end{problem}
	
	\begin{problem}{2.15}
		content...
	\end{problem}
\end{document}
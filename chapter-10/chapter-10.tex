\documentclass[10pt]{article}

\usepackage[margin=1in]{geometry} 
\usepackage{amsmath,amsthm,amssymb, graphicx, multicol, array}
\usepackage{tikz-cd} 

\newtheorem{innercustomgeneric}{\customgenericname}
\newtheorem{definition}{Definition}

\providecommand{\customgenericname}{}
\newcommand{\newcustomtheorem}[2]{%
	\newenvironment{#1}[1]
	{%
		\renewcommand\customgenericname{#2}%
		\renewcommand\theinnercustomgeneric{##1}%
		\innercustomgeneric
	}
	{\endinnercustomgeneric}
}

\newcustomtheorem{customthm}{Theorem}
\newcustomtheorem{customlemma}{Lemma}

\newcommand{\N}{\mathbb{N}}
\newcommand{\Z}{\mathbb{Z}}
\newcommand{\Q}{\mathbb{Q}}
\newcommand{\R}{\mathbb{R}}
\newcommand{\C}{\mathbb{C}}

\newenvironment{problem}[2][Problem]{\begin{trivlist}
		\item[\hskip \labelsep {\bfseries #1}\hskip \labelsep {\bfseries #2.}]}{\end{trivlist}}
	
\newenvironment{answer}[2][Answer]{\begin{trivlist}
		\item[\hskip \labelsep {\bfseries #1}\hskip \labelsep {\bfseries #2.}]}{\end{trivlist}}
	
\newenvironment{theorem}[2][Theorem]{\begin{trivlist}
		\item[\hskip \labelsep {\bfseries #1}\hskip \labelsep {\bfseries #2.}]}{\end{trivlist}}

\begin{document}
	
	\title{Exercises from Chapter 10}
	\author{Wesley Basener}
	\maketitle
	
	I am skipping the first few exercises of each section in this chapter because they didn't look challenging enough to be helpful.
	\begin{problem}{1.15}
		If $M$ is a finite abelian group then $M$ is naturally a $\Z$-module. Can this action be extended
		to make $M$ into a $\Q$-module?
		\begin{proof}
			No, consider an order two element $m \in M$. Supposing such a $\Q$-module were possible, multiplying $m$ by $\frac{2}{2} = (\frac{1}{2})(2)$ yields
			\begin{align*}
				\frac{2}{2}m = 1m = m \not= e = \frac{1}{2}e = \frac{1}{2}2m
			\end{align*}
			Which of course contradicts part 2-b of the definition of modules.
		\end{proof}
	\end{problem}
		
	\begin{problem}{1.17}
		Let $T$ be the shift operator on the vector space $V$ and let $e_1, ..., e_n$ be the usual basis vectors
		described in the example of $F[x]$-modules. If $m \geq n$ find $(a_m x^m +a_{m-1}x^{m-1} + ... + a_0 )e_n$.
		\begin{proof}
			Each $x^i$ shifts the $1$ in the basis over $i$ spaces left. Every $x^i$ where $i > n$ will leave $e_i = 0$. Hence, we have,
			\begin{align*}
				(a_m x^m +a_{m-1}x^{m-1} + ... + a_0 )e_n =
				(a_n, a_{n-1}, ..., a_0, 0, ..., 0)
			\end{align*} 
			where $a_0$ is in the $n$th spot.
		\end{proof}
	\end{problem}
	
	\begin{problem}{1.19}
		Let $F = \R$ , let $V = \R^2$ and let $T$ be the linear transformation from $V$ to $V$ which is projection onto the $y$-axis. Show that $V$, $0$, the $x$-axis and the $y$-axis are the only $F[x]$­
		submodules for this $T$.
		\begin{proof}
			Suppose for the sake of contradiction that this is not true, then there must exist a submodule $N$ with an element $(a,b)$ not in either of the nontrivial submodules mentioned. (ie. $a, b, \not = 0$). 
			
			Let $(c,d)$ be any other element of $M$ not in $N$. Multiplying $(a,b)$ by $(\frac{d}{b} - \frac{c}{a})x + \frac{c}{a}$ yields
			\begin{align*}
				((\frac{d}{b} - \frac{c}{a})x + \frac{c}{a})(a,b) =
				(0, d-\frac{cb}{a}) + (c,\frac{cb}{a}) =
				(c,d)
			\end{align*}
			Sending $(a,b)$ outside of its submodule $N$ is obviously a contradiction. Hence, there are no other non trivial submodules of $M$. In other words, $V$, $0$, the $x$-axis, and the $y$-axis are the only submodules of $M$ on this $T$.
		\end{proof}
	\end{problem}
	
	\begin{problem}{1.21}
		Let $n \in \Z^+, n > 1$ and let $R$ be the ring of $n \times n$ matrices with entries from a field $F$. Let
		$M$ be the set of $n \times n$ matrices with arbitrary elements of $F$ in the first column and zeros
		elsewhere. Show that $M$ is a submodule of $R$ when $R$ is considered as a left module over
		itself, but $M$ is not a submodule of $R$ when $R$ is considered as a right $R$-module.
		\begin{proof}
			Consider two elements of $M$: $a$ and $b$. Let $r$ be any element of $R$. If $R$ is a right module, we have
			\begin{align*}
				a + rb = \begin{pmatrix}
					a_{1,1} & \dots & 0 \\
					\vdots &  & \vdots \\
					a_{n,1} & \dots & 0
				\end{pmatrix}
				+
				\begin{pmatrix}
					r_{1,1} & \dots & r_{1,n} \\
					\vdots &  & \vdots \\
					r_{n,1} & \dots & r_{n,n}
				\end{pmatrix}
				\begin{pmatrix}
					b_{1,1} & \dots & 0 \\
					\vdots &  & \vdots \\
					b_{n,1} & \dots & 0
				\end{pmatrix} = 
				\begin{pmatrix}
					a_{1,1} + r_{1,*} \cdot b_{*,1} & \dots & 0 \\
					\vdots &  & \vdots \\
					a_{1,n} + r_{1,*} \cdot b_{*,1} & \dots & 0
				\end{pmatrix}
			\end{align*}
			which is in $M$. Hence $M$ is a submodule of $R$, when $R$ is a left module over itself.
			
			If we consider $R$ as a right module over itself, multiplication of any elememt $a$ in $M$ by any element $r$ not in $M$ yields.
			\begin{align*}
				ar = \begin{pmatrix}
					a_{1,1} & \dots & 0 \\
					\vdots &  & \vdots \\
					a_{n,1} & \dots & 0
				\end{pmatrix}
				\begin{pmatrix}
					r_{1,1} & \dots & r_{1,n} \\
					\vdots &  & \vdots \\
					r_{n,1} & \dots & r_{n,n}
				\end{pmatrix} = 
				\begin{pmatrix}
					a_{1,1}r_{1,1} & \dots & a_{1,1}r_{1,n} \\
					\vdots &  & \vdots \\
					a_{n,1}r_{1,1} & \dots & a_{n,1}r_{1,n}
				\end{pmatrix}
			\end{align*}
			which is not necessarily in $M$. Therefore, $M$ is not a submodule of $R$, when $R$ is a right module over itself.  
		\end{proof}
	\end{problem}
	
	\begin{problem}{3.1}
		Prove that if $A$ and $B$ are sets of the same cardinality, then the free modules $F(A)$ and
		$F(B)$ are isomorphic.
		\begin{proof}
			If $A$ and $B$ have the same cardinality, then there is a bijective map $\phi : A \rightarrow B$. We can extend this map to the function $\psi: F(A) \rightarrow F(B)$, with the definition $\psi(r_1a_1 + r_2a_2 + ... + r_na_n) = r_1\phi(a_1) + r_2\phi(a_2) + ... + r_n\phi(a_n)$. 
			
			This map is a homomorphis since $\psi(r_1a_1 + ... + r_na_n + r(r_1'a_1' + ... + r_m'a_m')) = r_1\phi(a_1) + ... + r_n\phi(a_n) + r(r_1'\phi(a_1') + ... + r_n'\phi(a_n'))$. 
			
			Furthermore, for unique $a_i$, $\psi(r_1a_1 + ... + r_na_n)=0$ would necessitate $\phi(a_i) = \phi(a_j)$ for some $j \not= i$, which is impossible since $\phi$ is a bijection. Hence, the kernel of the homomorphism is $0$ and it is thus an isomorphism. Therefore, $F(A)$ is isomorphic to $F(B)$.
		\end{proof}
	\end{problem}
	
	\begin{problem}{3.3}
		Show that the $F[x]$-modules in Exercises 18 and 19 of Section I are both cyclic.
		\begin{proof}
			In exercise 18, the $F[x]$-module is $\R^2$ with $T$ being the rotation about the origin by $\pi/2$ radians. The element $(1,0)$ can generate any other element $(a,b)$ with the polynomial $bx + a$. 
			
			In exercise 19, the $F[x]$-module is again $\R^2$ but this time with $T$ being the projection onto the $y$-axis. We showed in this exercise that any element $(a,b) \in \R^2$ not on the $y$ or $x$ axis can be mapped to any other element in $(c,d) \in \R^2$ with the polynomial $(\frac{d}{b} - \frac{c}{a})x + \frac{c}{a}$. Thus, any such element can generate the entire module.
		\end{proof}
	\end{problem}
	
	\begin{problem}{3.5}
		Let $R$ be an integral domain. Prove that every finitely generated torsion $R$-module has a
		nonzero annihilator i.e., there is a nonzero element $r \in R$ such that $rm = 0$ for all $m \in M$ here $r$ does not depend on $m$ (the annihilator of a module was defined in Exercise 9 of Section 1). Give an example of a torsion $R$-module whose annihilator is the zero ideal.
		\begin{proof}
			Let $\{m_1, ..., m_n\}$ be the generators for $M$ and let $t_1, ..., t_n \in R$ be the elements in $R$ such that $r_im_i=0$ for all $i$. Multiplying the product of the $t_i$ by any element in $M$ yields $(t_1...t_n)(r_1m_1 + ... + r_nm_n = (t_2...t_mr_1)t_1m_1 + ... + (t_1...t_{n-1}r_n)t_nm_n = 0$. Hence, $(t_1...t_n)$ is the desired element.
		\end{proof}
	\end{problem}
	
	\begin{problem}{3.7}
		Let $N$ be a submodule of $M$. Prove that if both $M/N$ and $N$ are finitely generated then so
		is $M$.
		\begin{proof}
			We have that the set of elements $x_1, x_2, ...$ such that $x_1 +N, x_2 + N, ...$ are unique elements in $M/N$ is finitely generated. Say, that this set is generated by $\{g_1, g_2, ... g_n\}$. Also say that $N$ is generated by $\{n_1, ... , n_m\}$. 
			
			Every element in $m \in M$ maps to a unique element in some set $x+N \in M/N$, for some $x \in M$. Say for example, $m = x' + n'$. Both $x'$ and $n'$ have unique generations $x' = \Sigma r_ig_i$ and $n' = \Sigma r_j n_j$. Hence, $m = \Sigma r_ig_i + \Sigma r_j n_j$ is a unique generation of $m$.
		\end{proof}
	\end{problem}
	
	\begin{problem}{3.9}
		An $R$-module $M$ is called irreducible if $M \not = 0$ and if $0$ and $M$ are the only submodules of M. Show that $M$ is irreducible if and only if $M \not= 0$ and $M$ is a cyclic module with any nonzero element as generator. Determine all the irreducible $Z$-modules.
		\begin{proof}
			Suppose $M$ is irreducible. Then for any two elements $m, n \in M$, with $m\not= 0$ there must be an $r \in R$ such that $rm = n$. If this were not so, $m$ would generate a non trivial submodule of $M$. Hence, any nonzeoro element generates $M$. We also have that $M\not=0$.
			
			Now, suppose $M \not = 0$ and that $M$ is cyclic, being generated by any nonzero element. Then for any elements $m, n \in M$ with $m \not= 0$ there is an $r \in R$ such that $rm=n$. In other words, we can send any nonzero $m \in M$ to any other element in $M$. Hence, no elements in $M$ are a part of a non trivial submodule.
			
			If $M$ is a irreducible $Z$-module, then any non zero $m$ in $M$ must generate $M$. This means there must be an integer $n$ such that $nm = m..(\text{n times})..m =e$. Thus, the irreducible $Z$ modules are those modules whose base group is cyclically generated by any non element item. 
		\end{proof}
	\end{problem}
	
	\begin{problem}{3.11}
		Show that if $M_1$ and $M_2$ are irreducible $R$-modules, then any nonzero $R$-module homomorphism from $M_1$ to $M2$ is an isomorphism. Deduce that if $M$ is irreducible then $\text{End}_R(M)$ is a division ring (this result is called Schur's Lemma). [Consider the kernel and the image.]
		\begin{proof}
			Let $\phi : M_1 \rightarrow M_2$ be a homomorphism from $M_1$ to $M_2$. Suppose that $m \in \text{ker}(\phi)$ and $m \not = 0$. From exercise 9, we know that any nonzero element of $M_1$ generates the entire module. So there must be an $r \in R$ such that $rm \not \in \text{ker}(\phi)$. But this is a contradiction since $0 \not = \phi(rm) = r\phi(m) = r0 = 0$. Hence the kernel of $\phi$ is precisely the element $0$ and $\phi$ is therefore an isomorphism.
			
			It has been established that $\text{End}_R(M)$ is a ring, with multiplication being defined as function composition. If $\psi \in \text{End}_R(M)$, then by the prior proof, $\psi$ is an isomorphism. Thus, it has an inverse $\psi^{-1}$. Multiplying these elements in $\text{End}_R(M)$ lends $\psi^{-1}\circ\psi =\psi\circ\psi^{-1} = 1$ the identity element on $\text{End}_R(M)$. Therefore, every element of $\text{End}_R(M)$ has a multiplicative inverse and it is a division ring.
		\end{proof}
	\end{problem}
	
	\begin{problem}{3.13}
		Let $R$ be a commutative ring and let $F$ be a free $R$-module of finite rank. Prove the following isomorphism of $R$-modules: $\text{Hom}_R(F, R) \cong F$.
		\begin{proof}
			 Define $\{a_0,...,a_n\}$ as the finite generators of F. Let $\phi \in \text{Hom}_R(F, R)$, and define a map $\pi$ between $\phi$ and $F$ as $\pi(\phi) = \phi(a_1)a_1 + ... + \phi(a_n)a_n$. This function is a homomorphism since \begin{align*}
			 	\pi(\psi + r\phi) = \psi(a_1) + ... + \psi(a_n) + r(\phi(a_1) + ... + \phi(a_n)) = \pi(\psi) + r\pi(\phi)
			 \end{align*}
			 The kernel of $\pi$ is also only the zero function of $\text{Hom}_R(F, R)$, since any other function would send some $a_i$ to a nonzero $r$. Therefore, $\pi$ is an isomorphism. 
		\end{proof}
	\end{problem}
	
	\begin{problem}{3.15}
		An element $e \in R$ is called a central idempotent if $e^2 = e$ and $er=re$ for all $r \in R$. If $e$ is a central idempotent in $R$, prove that $M = eM \oplus (1 -e)M$. [Recall Exercise 14 in Section 1.]
		\begin{proof}
			Exercise 14 asked us to prove that $zM$ is a submodule of $M$ for any $z$ in the center of $R$. This is done by noticing that $zm_1 + r(zm_2) = e(m_1 + rm_2)$. 
			
			We have that both $e$ and $1$ (so by extent $1-e$) are in the center of $M$. So $eM$ and $(e-1)M$ are both submodules of $M$.
			
			For any $r$ in $R$, either $e$ factors out of $r$, in which case $r(1-e)M = r'(e-e^2)M = r'(e-e)M = r'0M = 0$. Or, $e$ does not factor out of $r$, in which case $r(1-e)M \cap eM = \emptyset$. So, $(1-e)M$ and $eM$ only overlap at $0$.
			
			We can get any $m \in M$ from the direct sum $eM \oplus (1 -e)M$ with the element $em + (1-e)m = m$. Hence, $M = eM \oplus (1 -e)M$ spans $M$.
		\end{proof}
	\end{problem}
	
	\begin{problem}{3.17}
		In the notation of the preceding exercise, assume further that the ideals $A_t, ..., A_k$ are
		pairwise comaximal (i.e., $A_i + A_j = R$ for all $i \not= j$). Prove that
		$M/(A_1...A_k)M \cong M/A_1M\times...\times M/A_k M$. [See the proof of the Chinese Remainder Theorem for rings in Section 7.6.]
		\begin{proof}
			The previous exercise proved that the map $M \rightarrow M/A_1M \times ... \times M/A_kM$ defined by $m \rightarrow (m+A_1M,...,m+A_kM)$ is a homomorphism with kernel $A_1M\cap...\cap A_kM$.
			
			We can extend this map as one between $M/(A_1...A_k)M$ and $M/A_1M\times...\times M/A_k M$. Doing so preserves the same kernel. And, we find that the kernel $A_1M\cap...\cap A_kM$ is percisly the $0$ element of $M/(A_1...A_k)M$. Therefore, the homomorphism becomes an isomorphism between these two modules.
		\end{proof}
	\end{problem}
	
	\begin{problem}{3.19}
		Show that if $M$ is a finite abelian group of order $a = p_1^{\alpha_1} p_2^{\alpha_2} \dots p_k^{\alpha_k}$ then, considered as a $\Z$-module, $M$ is annihilated by $(a)$, the $p_i$-primary component of $M$ is the unique Sylow $p_i$-subgroup of $M$ and $M$ is isomorphic to the direct product of its Sylow subgroups.
		\begin{proof}
			Any element of $M$ must have order dividing $a$. Say, $m \in M$ has order $p$ and $p = ap'$. Then $p$ is in $(a)$ and $m^p = m^{p'}m^a = m^{p'}0 = 0$. Thus, $(a)$ annihilated by $(a)$.
			
			If $m$ is in the $p_i$-primary component of $M$, then $m^{p_i^{\alpha_i}}=0$. Thus, $m$ has order $p_i^{\alpha_i}$ and is in the Sylow $p_i$ subgroup of $M$.
			
			The Sylow subgroups of $M$ are pairwise disjoint. So, by proposition 5, $M$ is isomorphic to the direct product of its Sylow subgroups.
		\end{proof}
	\end{problem}
	
	\begin{problem}{3.21}
		Let $I$ be a nonempty index set and for each $i \in I$ let $N_i$ be a submodule of $M$. Prove that
		the following are equivalent:
		\begin{itemize}
			\item[\textbf{(i)}]the submodule of $M$ generated by all the $N_i$'s is isomorphic to the direct sum of the $N_i's$.
			\item[\textbf{(ii)}]if $\{i_1, i_2, ..., i_k\}$ is any finite subset of $I$ then $N_{i_1} \cap (N_{i_2} + \dots + N_{i_k}) = 0$
			\item[\textbf{(iii)}] if $\{i_1, i_2, ..., i_k\}$ is any finite subset of $I$ then $N_1 + \dots + N_k = N_1 \oplus \dots \oplus N_k$
			\item[\textbf{(iv)}] for every element $x$ of the submodule of $M$ generated by the $N_i$'s there are unique elements $a_i \in N_i$ for all $i \in I$ such that all but a finite number of the $a_i$ are zero and
			$x$ is the (finite) sum of the $a_i$.
		\end{itemize}
		\begin{proof}
			This is just proposition 5 with the added necessity of the axiom of choice. 
		\end{proof}
	\end{problem}
	
	\begin{problem}{3.23}
		Show that any direct sum of free $R$-modules is free.
		\begin{proof}
			Let $F_1, ..., F_k$ all be free $R$-modules over the respective sets $A_1, ..., A_k$. Any element of their direct product can be decomposed into $x_1 + ... + x_k$ where $x_i \in F_i$. We can further decompose this into $r_{1,1}a_{1,1} + ... + r_{1,m}a_{1,m} + ... + r_{k,1}a_{k,1} + ... + r_{k,n}a_{k,n}$, where $r_{i,j} \in R$ and $a_{i,j} \in A_i$. Hence, any element of the direct product of the free groups is free over the set $A_1 \cup ... \cup A_k$.
		\end{proof}
	\end{problem}
	
	\begin{problem}{3.25}
		In the construction of direct limits, Exercise 8 of Section 7.6, show that if all $A_i$ are $R$-modules and the maps $\rho_{ij}$ are $R$-module homomorphisms, then the direct limit $A = \text{lim}A_i$ may be given the structure of an $R$-module in a natural way such that the maps $\rho_i : A_i \rightarrow A$ are all $R$-module homomorphisms. Verify the corresponding universal property (part (e)) for $R$-module homomorphisms $\phi_i : A_i \rightarrow C$ commuting with the $\rho_{ij}$.
		\begin{proof}
			Let $a$ and $b$ be two elements of $A$. Define $ra$ as $\rho_{ij}(ra') = r\rho_{ij}(a')$ for all $a' \in a$ and define $a+b$ as $\rho_{ik}(a') + \rho_{ik}(b')$ for all $a' \in a$ and $b' \in b$ where $i,j \leq k$. The $R$-module structure of the $A_i$ ensures that we have defined a $R$-module on $A$.
			
			For any element $a$ in $A$, define the map $\phi:A \rightarrow C$ as $\phi(a) = \{a' | \phi_j \circ \rho_{ij} (a') = a\}$. This map is a homomorphism since
			\begin{align*}
				\phi(a + rb) = \{ x | \phi_j \circ \rho_{ij} (x) = a + rb\} = \{a' + rb' |  \phi_j \circ \rho_{ij} (a' + rb') = a + rb\} = \{a' | \phi_j \circ \rho_{ij} (a') = a\} + r\{b' | \phi_j \circ \rho_{kj} (b') = b\}
			\end{align*}
			for $r \in R$ and $a' \in a, b' \in b$. (//TODO make this more formal and fleshed out; prove uniqueness.)
		\end{proof}
	\end{problem}
	
	\begin{problem}{3.27}
		\textit{(Free modules over noncommutative rings need not have a unique rank)} Let $M$ be the
		$\Z$-module $Z \times Z \times ...$ of Exercise 24 and let $R$ be its endomorphism ring, $R= \text{End}_{\Z}(M)$	(cf. Exercises 29 and 30 in Section 7.1). Define $\phi_1, \phi_2 \in R$ by
		\begin{align*}
			\phi_1(a_1, a_2, a_3, ...) = (a_1, a_3, a_5,...)
		\end{align*}
		\begin{align*}
			\phi_1(a_1, a_2, a_3, ...) = (a_2, a_4, a_6,...)
		\end{align*}
		\begin{itemize}
			\item[\textbf{(a)}]Prove that $\{\phi_1 , \phi_2\}$ is a free basis of the left $R$-module $R$. [Define the maps $\psi_1$ and $\psi_2$ bY $\psi_1(a_1, a_2, ...) = (a_1, 0, a_2, 0, ...)$ and $\psi_2(a_1, a_2, ...) = (O, a_1 , 0, a_2, ...)$. Verify that $\phi_i\psi_i = 1, \phi_1\psi_2 = 0 = \phi_2\psi_1$ and $\psi_1\phi_1 + \psi_2\phi_2 = 1$. Use these relations to prove that $\phi_1, \phi_2$ are independent and generate $R$ as a left $R$-module.]
			\item[\textbf{(b)}] Use (a) to prove that $R \cong R^2$ and deduce that $R \cong R^n$ for all $n \in \Z^+$.
		\end{itemize}
	\end{problem}
	
	\begin{definition}{Tensor Products}
		Let $S$ be a ring and $N$ be a left $R$-module. In the free group $F(S\times N)$, define the subgroup $H$ of the free group as all elements of the form $(s_1+s_2,n)-(s_1,n)-(s_2,n), (s,n_1+n_2)-(s,n_1)-(s,n_2), (sr,n)-(s,rn)$ where $s, s_1, s_2 \in S, n, n_1, n_2 \in N, r \in R$. Then the tensor product $S \otimes_R N$ is defined as $F(S\times N)/H$
	\end{definition}
	
	\begin{theorem}{8}
		Let $R$ be a subring of $S$, let $N$ be a left $R$-module and let $\iota : N \rightarrow S \otimes_R N$ be the $R$-module homomorphism defined by $\iota(n) = 1\oplus n$. Suppose that $L$ is any left $S$ module (hence also an $R$-module) and that $\varphi: N \rightarrow L$ is an $R$-module homomorphism	from $N$ to $L$. Then there is a unique $S$-module homomorphism $\phi : S \otimes_R N \rightarrow L$ such	that $\phi$ factors through $\phi$, i.e., $\phi = \varphi \circ \iota$ and the diagram
		\[ \begin{tikzcd}
			N \arrow[rd, "\varphi"'] \arrow{r}{\iota} & S\otimes_RN \arrow{d}{\phi} \\%
			& L
		\end{tikzcd}
		\]
		commutes. Conversely, if $\phi : S \otimes_R N \rightarrow L$ is an $S$-module homomorphism then $\phi = \iota \circ \varphi$ is an $R$-module homomorphism from $N$ to $L$.
		\begin{proof}
			
		\end{proof}
	\end{theorem}
	
	\begin{problem}{4.1}
		Let $f:R \rightarrow S$ be a ring homomorphism from the ring $R$ to the ring $S$ with $f(1_R)=1_S.$ Verify the details that $sr=sf(r)$ defines a right $R$-action on $S$ under which $S$ is an $(S,R)$-bimodule.
		\begin{proof}
			In virtue of $S$ being a ring, we already have that $sr=sf(s) \in S$ is a left $S$-module. So we only need to show that it is also a right $R$-module by proving part $2$ of the definition of modules on page 337.
			
			For a, we have that $s(r_1 + r_2) = sf(r_1 + r_2) = sf(r_1) + sf(r_2) = sr_1 + sr_2$. For part b, we have $s(r_1r_2) = sf(r_1r_2) = sf(r_1)f(r_2) = s(r_1)(r_2)$. For c, we have $(s_1 + s_2)r = (s_1 + s_2)f(r) = s_1f(r) + s_2f(r)$. Lastly, for part d, we have $s1_R = sf(1_R) = s1_S = s$. Together, these conditions imply that the set is a right $R$-module and is hence an $(S,R)$-bimodule.
		\end{proof}
	\end{problem}
	
	\begin{problem}{4.3}
		Show that $\C \otimes_\R \C$ and $\C \otimes_\C \C$ are both left $\R$-modules but are not isomorphic as $\R$-modules.
		\begin{proof}
			In virtue of the fact that $\C$ is a left $\R$-module, both tensor products are left $\R$-modules.
			
			Suppose $\phi$ is an isomorphism between $\C \otimes_\R \C$ and $\C \otimes_\C \C$. Then $\phi((1,i)-(i,1)) = (1,i) - (i,1) = (1,i)-(1,i)=0$. Which is a contradiction because on $\C \otimes_\R \C$, $(1,i)-(i,1) \not = 0$. So, there is no such isomorphism.
		\end{proof}
	\end{problem}
	
	\begin{problem}{4.5}
		Let $A$ be a finite abelian group of order $n$ and let $p^k$ be the largest power of the prime $p$ dividing $n$. Prove that $\Z/p^k\Z \otimes_\Z A$ is isomorphic to the Sylow $p$-subgroup of $A$.
		\begin{proof}
			Let $\phi$ be the homomorphism from $A$ to $\Z/p^k\Z \otimes_\Z A$ defined by $\phi(a) = 1\otimes a$.
			
			Let $a$ be an element of $A$ and $|a|=m$. If $a$ is not in the Sylow $p^k$-subgroup of $A$, then  and $(p^k, m) = 1$. So, by Bezout's lemma, there are numbers $x$ and $y$ such that $mx+p^ky = 1$. Therefore, $\phi(a) = 1\otimes a = mx \otimes a = 1 \otimes xma = 1 \otimes 0$.
			
			On the other hand, if $a$ is in the Sylow $p^k$-subgroup of $A$, then $(p^k, m)=m$, then $1 \otimes a \not = 0$. 
			
			Thus, $a$ is in the kernel of the homomorphism, if and only if $a$ is not in the Sylow $p^k$-subgroup of $A$. So we have $\Z/p^k\Z \otimes_\Z A \cong A/\text{ker}(\phi)$, which is of course isomorphic to the Sylow $p^k$-subgroup of $A$.
		\end{proof}
	\end{problem}
	
	\begin{problem}{4.7}
		If $R$ is any integral domain with quotient field $Q$ and $N$ is a left $R$-module, prove that every element of the tensor product $Q \otimes_R N$ can be written as a simple tensor of the form $(1/d) \otimes n$ for some nonzero $d \in R$ and some $n \in N$.
		\begin{proof}
			Let $a/d \otimes b$ be any element of $Q \otimes_R N$, then we can use the distributive property of tensor products to yield
			\begin{align*}
				a/d \otimes b = 1/d \otimes ab
			\end{align*}
			Which is obviously the desired format. Since every element of $Q \otimes_R N$ has the form $a/d \otimes b$, we are done.
		\end{proof}
	\end{problem}
	
	\begin{problem}{4.9}
		Suppose $R$ is an integral domain with quotient field $Q$ and let $N$ be any $R$-module. Let $Q \otimes_R N$ be the module obtained from $N$ by extension of scalars from $R$ to $Q$. Prove that the $R$-module homomorphism $\iota : N \rightarrow Q \otimes_R N$ is the torsion submodule of $N$.
		\begin{proof}
			The torsion submodule of $N$ is the submodule formed by all $n \in N$ such that $rn =0$ for some $r \in R$.
			
			Let $n$ be any element of $N$. The equivalence class of all elements in $\iota(n) = 1 \otimes n$ is every element of the form $\frac{1}{r} \otimes rn$ for all $r \in R$. Since $R$ is an integral domain, $\frac{1}{r}$ will never be $0$. Thus, the tensor is $0$ if and only if $rn =0$. Therefore, the kernel of the function is precisely the torsion elements of $N$.
		\end{proof}
	\end{problem}
	
	\begin{problem}{4.11}
		
	\end{problem}
	
	\begin{problem}{4.15}
		Show that tensor products do not commute with direct products in general. [Consider
		the extension of scalars from $\Z$ to $\Q$ of the direct product of the modules $M_i = \Z/2^i\Z, i=1,2,...]$
		\begin{proof}
			Essentially, we want to show that for $R$-modules $M_1$ and $M_2$, and $R$ being a subring of $N$, the statement $N \otimes_R (M_1 \oplus M_2) \cong (N \otimes_R M_1) \oplus (N \otimes_R M_2)$ is not always true.
			
			Consider the direct tensor extension of the product $\Q \otimes_\Z M_1 \oplus M_2 \oplus ...$ where $M_i = \Z/2^i\Z, i=1,2,...]$. If the element $1 \otimes 1 \oplus 1 \oplus 1...$ is zero, then there must be an integer $z$ such that $z1 = 0 \text{mod}(2^i)$ for all $i$ so that we woulld have $\frac{1}{z} \otimes z \oplus z \oplus ... = 1 \otimes 0 \oplus 0 \oplus$. However, no such integer exists because obviously all integers are divisible by a finite power of $2$. Hence, $\Q \otimes_\Z M_1 \oplus M_2 \oplus ...$ has a nonzero element.
			
			Now consider the module $(\Q \otimes_\Z M_1) \oplus (\Q \otimes_\Z M_2) \oplus ...$, which is the same module as before, except we exchange the order of operations of the tensor extension and the direct product. Notice however that for any element $q \otimes m$ each $\Q \otimes_\Z M_i$, $q \otimes m = \frac{q}{2^i} \otimes 2^im = 1 \otimes 0$. Hence, the entire module is the direct product of zero modules and is thus the zero module.
			
			We have shown that switching the order of operations on the tensor product produces different modules. Therefore, these operations do not commute.
		\end{proof}
	\end{problem}
	
	\begin{problem}{4.17}
		Let $I = (2, x)$ be the ideal generated by $2$ and $x$ in the ring $R = \Z[x]$. The ring $\Z/2\Z = R/I$	is naturally an $R$-module annihilated by both $2$ and $x$.
		\begin{itemize}
			\item[\textbf{(a)}] Show that the map $\phi : I \times I \rightarrow \Z/2\Z$ defined by
			\begin{align*}
				\phi(a_0 + a_1x + ... + a_nx^n, b_0 + b_1x + ... b_mx^m) = \frac{a_0}{2}b_1 \text{mod}2
			\end{align*}
			is $R$-bilinear.
			\item[\textbf{(b)}] Show that there is an $R$-module homomorphism from $I \otimes_R I \rightarrow \Z/2\Z$ mapping $p(x)\otimes q(x)$ to $\frac{p(0)}{2}q'(0)$ where $q'$ denotes the usual polynomial derivative of $q$.
			\item[\textbf{(c)}] Show that $2 \otimes x \not = x \otimes 2$ in $I \otimes_R I$.
		\end{itemize}
		\begin{proof}{(a)}
			Since $2$ and $x$ are prime, $I$ is a prime ideal. Any factorization of elements in $I$ will yield at least one element in $I$.
			
			With that in mind, let $q(x)$ be in $R$ and $a(x), b(x)$ be in $I$ with leading terms $q_0, a_0$ and $b_0$ respectively. Multiplying $q(x)$ by the first term and entering it into the function yields
			\begin{align*}
				\phi(q(x)a(x),b(x)) = \frac{q_0a_0}{2}b_1 \text{mod} 2 = q_0(\frac{a_0}{2}b_1 \text{mod} 2)
			\end{align*}
			
			We again enter the polynomials into the function, this time multiplying $q(x)$ by the second term yielding
			\begin{align*}
				\phi(a(x), q(x)b(x)) = \frac{a_0}{2}(q_0b_1 + q_1b_0) \text{mod} 2 = \frac{a_0}{2}(q_1b_0) \text{mod} 2 + \frac{a_0}{2}(q_0b_1) \text{mod} 2
			\end{align*}
			Since $b_0 \in I$, we can factor the first term in the sum out as $0$ resulting in
			\begin{align*}
				\frac{a_0}{2}q_0b_1 \text{mod} 2 = q_0(\frac{a_0}{2}b_1 \text{mod} 2)
			\end{align*}
			Since $\Sigma_{i=1}^{\text{deg}(q(x))}q_ix^i$ is in $I$, it zeros out on the module $\Z/2\Z$. Hence, 
			\begin{align*}
				q_0(\frac{a_0}{2}b_1 \text{mod} 2) = q_0(\frac{a_0}{2}b_1 \text{mod} 2) + \Sigma_{i=1}^{\text{deg}(q(x))}q_ix^i(\frac{a_0}{2}b_1 \text{mod} 2) = q(x)(\frac{a_0}{2}b_1 \text{mod} 2)
			\end{align*}
			
			
			If we have $a(x) = b(x) + c(x)$ and $d(x) = e(x) + f(x)$ with all polynomials in $I$, then 
			\begin{align*}
				\phi(a(x), b(x)) = \frac{a_0}{2}b_1 \text{mod} 2 = \frac{b_0 + c_0}{2}(d_1) \text{mod} 2 =  \frac{b_0}{2}(d_1) \text{mod} 2 +  \frac{c_0}{2}(d_1) \text{mod} 2 = \phi(b(x), d(x)) + \phi(c(x), d(x))
			\end{align*}
			and
			\begin{align*}
				\phi(a(x), b(x)) = \frac{a_0}{2}b_1 \text{mod} 2 = \frac{a_0}{2}(e_1 + f_1) \text{mod} 2 =  \frac{a_0}{2}(e_1) \text{mod} 2 +  \frac{a_0}{2}(f_1) \text{mod} 2 = \phi(a(x), e(x)) + \phi(a(x), f(x))
			\end{align*}
			We have shown that $\phi$ is factorable over elements of $R$ and that $\phi$ is separable over sums in $I$. Hence, it is bilinear.
		\end{proof}
		\begin{proof}[(b)]
			As we have just shown, $\phi$ is bilinear. Hence, it is a $R$-module homomorphism. We note that $p(0) = p_0$ and $q'(0) = q_1$. Therefore, $\phi(p(x), q(x)) = \frac{p_0}{2}q_1 \text{mod} 2 = \frac{p(0)}{2}q'(0) \text{mod} 2$. So $\phi$ has the desired affect, but acts on $I \times I$ and not $I \otimes I$. Corollary 12 implies that there exists some homomorphism $\phi'$ such that $\phi = \phi' \circ \iota$. The homomorphism $\iota$ is an inclusion, and thus it does not change any actual function values apart from mapping elements to their corresponding place in its range. Therefore, $\phi'$ is the desired homomorphism.
		\end{proof}
		\begin{proof}[(c)]
			If this were true, the the homomorphism $\phi'$ from part (b) would have $2\otimes x - x\otimes 2$ in its kernel. However, $\phi'(2, x) - \phi(x, 2) = \frac{2}{2}1 \text{mod} 2 - \frac{0}{2}0 \text{mod} 2 = 1$. Hence, $2\otimes x - x\otimes 2$ is not in the kernel of $\phi'$. Therefore, these elements are not equal.
		\end{proof}
	\end{problem}
	
	\begin{problem}{4.19}
		Let $I = (2, x)$ be the ideal generated by $2$ and $x$ in the ring $R = \Z[x]$ as in Exercise 17. Show that the nonzero element $2 \otimes x - x \otimes 2$ in $I \otimes_R I$ is a torsion element. Show in	fact that $2 \otimes x - x \otimes 2$ is annihilated by both $2$ and $x$ and that the submodule of $I \otimes_R I$ generated by $2 \otimes x - x \otimes 2$ is isomorphic to $R/I$.
		\begin{proof}
			Multiplying the element by $2$ yields $2(2 \otimes x - x \otimes 2) = 2 \otimes 2x - 2x \otimes 2 = 2 \otimes 2x - 2 \otimes 2x = 0$. Similarly, multiplying by $x$ results in $x(2 \otimes x - x \otimes 2) = 2x \otimes x - x \otimes 2x = 2x \otimes x - 2x \otimes x = 0$. Hence, the element is annihilated by $2$ and $x$. 
			
			Any element of $I$ multiplied by $2 \otimes x - x \otimes 2$ results in $0$. Consider multiplication of the tensor sum by some $r \not \in I$. $r(2 \otimes x - x \otimes 2) = r2 \otimes x - rx \otimes 2$. There is no way to reduce any individual side of the tensor products to the product of elements in $I$, so we are not able to move $x$ or $2$ to a different side of the product without leaving the original side as a non element of $I$. Hence, I is precisely the annihilator of the tensor sum.
			
			Define $\phi$ as the map from $I \otimes_R I$ to $R/I$ as $\phi(r(2 \otimes x - x \otimes 2)) = r$. If $r$ is an element of $I$, the $\phi(r)=0$, otherwise it acts as the identity function to $r$. Hence, $\text{ker} = I$. By the first homomorphism theorem on modules $\phi$ is a function to $R/I$.
		\end{proof}	
	\end{problem}
	
	\begin{problem}{4.21}
		Suppose R i s commutative and let I and J be ideals of R .
		\begin{itemize}
			\item[\textbf{(a)}] Show there is a surjective $R$-module homomorphism from $I \otimes_R J$ to the product ideal $IJ$ mapping $i \otimes j$ to the element $ij$.
			\item[\textbf{(b)}] Give an example to show that the map in (a) need not be injective (cf. Exercise 17).
		\end{itemize}
		\begin{proof}{(a)}
			Any element of $IJ$ has the form $ij$ where $i \in I$ and $j\in J$. This can be mapped to the, not necessarily unique, element $i \otimes j$ in $I \otimes_R J$.
		\end{proof}
		\begin{proof}{(b)}
			IDK
		\end{proof}
	\end{problem}
	
	\begin{problem}{4.23}
		Verify the details that the multiplication in Proposition 19 makes $A \otimes_R B$ into an $R$-algebra. (Note, there is a typo in this question; it should say Proposition 21).
		\begin{proof}
			The proof in the book already shows that $A \otimes_R B$ has well defined linear multiplication by $R$. We now need to show that the set $A \otimes_R B$ has ring structure on the given multiplicative operation. Then, $A \otimes_R B$ will be a $R$-modules and a commutative ring, hence a $R$-algebra.
			
			Starting with addition, it is clear that for any $a, a' \in A$ and $b, b' \in B$, $a\otimes b + a' \otimes b'$ is an element of $A \otimes_R B$ regardless of whether the expression reduces to a single simple tensor. Hence, addition is closed. We also have that $0 \otimes 0$ is the identity element with addition, since $a \otimes b + 0 \otimes 0 = a \otimes b + 0\cdot0 \otimes b = a + 0 \otimes b = a \otimes b$. For any element, $a \otimes b$, addition by $a \otimes -b = -a \otimes b$ yields $a \otimes b + -a \otimes b = a-a \otimes b = 0 \otimes 1$. Thus, addition has an inverse operation.
			
			On multiplication, the book already showed that the operation is well defined, closed, and distributive. Therefore, $A \otimes_RB$ is a ring on these operations. Taken together with its modular properties over $R$, we have shown that $A\otimes_R B$ is a $R$-algebra.
		\end{proof}
	\end{problem}
	
	\begin{problem}{4.25}
		Let $R$ be a subring of the commutative ring $S$ and let $x$ be an indeterminate over $S$. Prove that $S[x]$ and $S \otimes_R R[x]$ are isomorphic as $S$-algebras.
		\begin{proof}
			Define multiplication and addition on $S \otimes_R R[x]$ as in proposition 21. Define the function $\phi S \otimes_R R[x] \rightarrow S[x]$ as $\phi(s \otimes r(x)) = sr(x)$ and $\phi(s\otimes r(x) + s' \otimes r'(x)) = \phi(s\otimes r(x)) + \phi(s' \otimes r'(x))$
			
			By proposition 21, we have bilinearity on $S \otimes_R R[x]$. So for any $a \in R$, $\phi(a(s \otimes r(x))) = \phi(as \otimes r(x)) = asr(x) = a\phi(s \otimes r(x))$. Thus, the function is linear on $R$.
			
			If $s' s \in S$ and $r' r \in R[x]$, then $\phi(s \otimes r(x) + s' \otimes r(x)) = \phi(s \otimes r(x)) + \phi(s' \otimes r(x)) = sr(x) + s'r(x) = (s+s')r(x) = \phi(s+s' \otimes r(x))$. A similar line of reasoning shows that $\phi$ is separable in the second variable.
			
			On multiplication, we have $\phi((s\otimes r(x))(s'\otimes r'(x)) = \phi((ss'\otimes r(x)r'(x)) = ss'rr'(x) = sr(x)s'r'(x) = \phi(s \otimes r)\phi(s \otimes r)$.
			
			We have thus shown that $\phi$ is a ring and module homomorphism, hence a $R$-algebra homomorphism. To see that it is an isomorphism, suppose $\phi(s \otimes r(x)) = 0$ for some $s \otimes r(x)$. Then, splitting $r(x)$ into the terms $r_0 + r_1x + r_2x + ...$ and noting that $sr_i=0$ for all $i$, we have $s \otimes r(x) = s \otimes r_0 + s\otimes r_1x + ... = 1 \otimes sr_0 + 1 \otimes sr_1 + ... = 0$, assuming $S$ has unity, which we are given. Thus, the kernel of this homomorphism is precisely the $0$ element of $S \otimes_R R[x]$. Therefore, $\phi$ is an isomorphism and $S \otimes_R R[x] \cong S[x]$.
		\end{proof}
	\end{problem}
	
	\begin{theorem}{30} Let $P$ be an $R$-module. Then the following are equivalent:
		\begin{itemize}
			\item[\textbf{(1)}] 
			For any $R$-modules $L,M,$ and $N$, if
			\[ \begin{tikzcd}
				0 \arrow{r} & L \arrow{r}{\psi} & M \arrow{r}{\varphi} & N \arrow{r} & 0
			\end{tikzcd}
			\]
			is a short exact sequence, then
			\[ \begin{tikzcd}
				0 \arrow{r} & \text{Hom}_R(P,L) \arrow{r}{\psi'} & \text{Hom}_R(P,M) \arrow{r}{\varphi'} & \text{Hom}_R(P,N) \arrow{r} & 0
			\end{tikzcd}
			\]
			is also a short exact sequence
			
			\item[\textbf{(2)}]
			For any $R$-modules $M$ and $N$, if 
			\begin{tikzcd}
				M \arrow{r}{\varphi} & N \arrow{r} & 0
			\end{tikzcd}
			is exact, then every $R$-module homomorphism from $P$ into $N$ lifts to a $R$-module homomorphism into $M$, i.e., given $f \in \text{Hom}_R(P,N)$ there is a lift $F \in \text{Hom}_R(P,M)$ making the following diagram commute:
			\[ \begin{tikzcd}
				& P \arrow{ld}{F} \arrow{d}{f}\\%
				M \arrow{r}{\varphi} & N \arrow {r} & 0
			\end{tikzcd}
			\]
			
			\item[\textbf{(3)}]
			If $P$ is a quotient of the $R$-module $M$ then $P$ is isomorphic to a direct summand of $M$, i.e., every short exact sequence
			\begin{tikzcd}
				0 \arrow{r} & L \arrow{r}{\alpha} & M \arrow{r}{\beta} & P \arrow{r} & 0
			\end{tikzcd}
			splits
			
			\item[\textbf{(4)}]
			$P$ is the direct summand of a free $R$-module
		\end{itemize}
		\begin{proof}
			Suppose 1 is true. Since $\varphi$ is surjective, the image of $f$ has an isomorphic preimage in $M$. So we can define $F$ to be the function carrying the preimage of $f$ to its image on $f$ then to its $\varphi$ preimage on $M$. This action is undone by $F$, hence, for all $p \in P$, $\varphi(F(p)) = f(p)$. Suppose 2 is true. then since 
			\begin{tikzcd}
				M \arrow{r}{\beta} & P \arrow{r} & 0
			\end{tikzcd}
			is exact. We can set $f$ as the isomorphism from $P$ to itself, and we can let $F \circ \beta = f$ hence, the preimage of $\beta$ in $M$ is isomorphic to $P$. Every element outside of the preimage of $\beta$ is in the kernel of $\beta$, and is thus in the image of $\alpha$. Thus, $M = L \oplus P$, meaning that the sequence splits. Now, suppose we are given 3. Let $\mathcal{F}$ be the free module formed by the elements of $P$. Let $\mathcal{A}$ be the submodule of $\mathcal{P}$ formed by all elements that are $0$ in $P$. Consider the diagram
			\begin{tikzcd}
				0 \arrow{r} & \mathcal{A} \arrow{r}{\iota} & \mathcal{P} \arrow{r}{\varphi} & \mathcal{P}/\mathcal{A} \arrow{r} & 0
			\end{tikzcd}
			where $\iota$ is the inclusion homomorphism and $\varphi$ is the natural homomorphism. Noticing that $\mathcal{P}/\mathcal{A} \cong P$, we have that a complement of $P$ is contained in the free group $\mathcal{P}$. Hence, $P$ is a summation of free modules. Finally, suppose we are given 4...
		\end{proof}
	\end{theorem}
	
	\begin{problem}{5.1}
		Suppose that
		\[ \begin{tikzcd}
			A \arrow{d}{\alpha} \arrow{r}{\psi} & B \arrow{d}{\beta} \arrow{r}{\varphi} & C \arrow{d}{\gamma} \\%
			A' \arrow{r}{\psi'} & B' \arrow{r}{\varphi'} & C'
		\end{tikzcd}
		\]
		is a commutative diagram of groups and that the rows are exact. Prove that
		\begin{itemize}
			\item[\textbf{(a)}] if $\varphi$ and $\alpha$ are surjective, and $\beta$ is injective then $\gamma$ is injective. [If $c \in \text{ker}\gamma$, show there is $a b \in B$ with $\varphi(b) = c$. Show that $\phi'(\beta(b)) = 0$ and deduce that $\beta(b) = \psi'(a')$ for some $a' \in A'$. Show there is an $a \in A$ with $\alpha(a) = a'$ and that $\beta(\psi(a)) = \beta(b)$. Conclude that $b = \psi(a)$ and hence $c = \phi(b) = 0$.]
			\item[\textbf{(b)}] if $\psi', \alpha,$ and $\gamma$ are injective then $\beta$ is injective.
			\item[\textbf{(c)}] if $\varphi, \alpha,$ and $\gamma$ are surjective then $\beta$ is surjective.
			\item[\textbf{(d)}] if $\beta$ is injective, $\alpha$ and $\varphi$ are surjective, then $\gamma$ is injective.
			\item[\textbf{(e)}] if $\beta$ is surjective, $\gamma$ and $\psi'$ are injective, then $\alpha$ is surjective.
		\end{itemize}
		\begin{proof}{(a)}
			This statement is equivalent to the kernel of $\gamma$ being $0$. Suppose $c \in \text{ker}(\gamma)$. Since $\varphi$ is surjective, there is some $b \in B$ such that $\varphi(b) = c$. Since $\beta$ is injective, there is only one element $b' \in B'$ such that $\beta(b) = b'$. Since the diagram commutes, $\gamma(\varphi(b)) = 0 = \beta(\varphi'(b))$. Hence, $b'$ is in the kernel of $\varphi'$. From the exactness of the rows, we have that there is some $a' \in A'$ such that $\psi'(a') = b'$. The function $\alpha$ is surjective, so there must be some $a \in A$. such that $\alpha(a) = a'$. Again, the commutativity gives us $\psi'(\alpha(a)) = b' = \beta(\psi(a))$. Thus, $b$ is in the image of $\psi$, and is hence in the kernel of $\varphi$. Therefore, $c = \varphi(b) = 0$, concluding the proof.
		\end{proof}
		\begin{proof}[(b)]
			This statement is equivalent to the kernel of $\beta$ being $0$. Let $b$ be in the kernel of $\beta$. Commutativity ensures that $\varphi'(\beta(b)) = 0 = \gamma(\varphi(b))$. Since $\gamma$ is injective, $\varphi(b) = 0$ and $b$ is in the image of $\psi$. Let $a \in A$ be an element such that $\psi(a) = b$. Then, $\beta(\psi(a)) = 0 = \psi'(\alpha(a))$. Since $\psi'$ and $\alpha$ are both injective, $a = 0$. Hence, $b = \psi(a) = 0$, finishing the proof.
		\end{proof}
		\begin{proof}[(c)]
			Let $b'$ be any element of $B'$. If $b'$ is in the kernel of $\varphi'$, then it is in the image of $\psi'$. Since $\alpha$ is surjective, there is some element $a$ in $\alpha$ such that $\psi'(\alpha(a)) = b'$. By commutativity, we have that $b' = \psi(\alpha(a)) = \beta(\psi(a))$. Hence $b'$ is in the image of $\beta$. If $b'$ is not in the kernel of $\varphi'$, then $\varphi'(b') = c \not = 0$. By the surjectivity of $\gamma$ and $\varphi$, we have $\gamma(\varphi(b)) = c$ for some $b \in B$. Again, by commutativity, we must have $\varphi'(\beta(b)) = c$ and $\beta(b) = b'$. Therefore, $b'$ is in the image of $\beta$ and $\beta$ is surjective.
		\end{proof}
		\begin{proof}[(d)]
			Let $c$ be in the kernel of $\gamma$. Since $\varphi$ is surjective, there is a $b \in B$ such that $\varphi(b) = c$. Because $\beta$ is injective, there is exactly one element $b' \in B'$ where $\beta(b)=b'$. By commutativity, $\gamma(\varphi(b)) = 0 = \varphi'(\beta(b))$. Thus, $\beta(b) = b'$ is in the kernel of $\phi'$ and is hence in the image of $\psi'$. So, for some $a'\in A'$, we have $\psi'(a')=b'$. As $\alpha$ is surjective, there is some $a \in A$ such that $\alpha(a) = a'$. Again, by commutativity, $\psi'(\alpha(a)) = \beta(\psi(a))$ hence, $a$ is in the image of $\psi$ and is therefore in the kernel of $\varphi$. Therefore, $c = \varphi(\psi(a)) = 0$ and $\gamma$ is thus injective.  
		\end{proof}
		\begin{proof}[(e)]
			Let $a'$ be any element of $A'$. Then $\psi'(a')$ equals exactly one element $b'$ in $B'$, since $\psi'$ is injective. Hence, $\varphi'(b') = 0$. Since $\beta$ is surjective, there is a $b$ in $B$ such that $\beta(b) = b'$. By commutativity, $\varphi'(\beta(b)) = 0 = \gamma(\varphi(b))$. Because $\gamma$ is injective, $\varphi(b) = 0$. Hence, $b$ is in the image of $\psi$, for some $a \in A$. Again, by commutativity, we have $\beta(\psi(a)) = b' = \psi'(\alpha(a))$. And, since $\psi'$ is injective, $\alpha(a) = a'$. Therefore, $\alpha$ is surjective.
		\end{proof}
	\end{problem}
	
	\begin{problem}{5.3}
		Let $P_1$ and $P_2$ be $R$-modules. Prove that $P_1 \oplus P_2$ is a projective $R$-module if and only if both $P_1$ and $P_2$ are projective.
		\begin{proof}
			Suppose for any $R$-Modules $M, L,$ and $N$ in a short exact sequence,
			\[ \begin{tikzcd}
				0 \arrow{r} & \text{Hom}_R(P_1,L) \arrow{r}{\psi_1} & \text{Hom}_R(P_1,M) \arrow{r}{\varphi_1} & \text{Hom}_R(P_1,Nd) \arrow{r} & 0
			\end{tikzcd}
			\]
			and
			\[ \begin{tikzcd}
				0 \arrow{r} & \text{Hom}_R(P_2,L) \arrow{r}{\psi_2} & \text{Hom}_R(P_2,M) \arrow{r}{\varphi_2} & \text{Hom}_R(P_2,N) \arrow{r} & 0
			\end{tikzcd}
			\]
			are both also short exact sequences. Then by proposition 22. $\psi_1$ and $\psi_2$ are both injective and $\phi_1$ and $\phi_2$ are both surjective. Therefore, the combined function $(\psi_1, \psi_2) : \text{Hom}(P_1,L)\oplus \text{Hom}_R(P_2,L) \rightarrow \text{Hom}(P_1,M)\oplus \text{Hom}_R(P_2,M)$ and $(\varphi_1, \varphi_2) : \text{Hom}(P_1,M)\oplus \text{Hom}_R(P_2,M) \rightarrow \text{Hom}(P_1,N)\oplus \text{Hom}_R(P_2,N)$ are also injective and surjective respectively.
			
			By proposition 29, $\text{Hom}_R(P_1 \oplus P_2, M) \cong \text{Hom}(P_1,M)\oplus \text{Hom}_R(P_2,M)$. Hence, $(\phi_1, \phi_2)$ and $(\psi_1, \psi_2)$ are both injective and surjective functions and the sequence
			\[ \begin{tikzcd}
				0 \arrow{r} & \text{Hom}_R(P_1 \oplus P_2,L) \arrow{r}{(\psi_1, \psi_2)} & \text{Hom}_R(P_1 \oplus P_2,M) \arrow{r}{(\varphi_1, \varphi_2)} & \text{Hom}_R(P_1 \oplus P_2,N) \arrow{r} & 0
			\end{tikzcd}
			\]
			is short exact. The converse proof follows similarly.
			
			A better way to prove this, which I realized after reading theorem 31, is that $P_1 \oplus P_2$ is the direct sum of free modules if and only if $P_1$ and $P_2$ are both direct summands of free $R$-modules. The proof follows directly from corollary 3.
		\end{proof}
	\end{problem}
	
	\begin{problem}{5.5}
		Let $A_1$ and $A_2$ be $R$-modules. Prove that $A_1 \oplus A_2$ is a flat $R$-module if and only if both $A_1$ and $A_2$ are flat. More generally, prove that an arbitrary direct sum $\Sigma A_i$ of $R$-modules is flat if and only if each $A_i$ is flat. [Use the fact that tensor product commutes with arbitrary direct sums.]
		\begin{proof}
			
		\end{proof}
	\end{problem}
\end{document}
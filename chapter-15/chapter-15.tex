\documentclass[10pt]{article}

\usepackage[margin=1in]{geometry} 
\usepackage{amsmath,amsthm,amssymb, graphicx, multicol, array}
\usepackage{tikz-cd} 

\newcommand{\N}{\mathbb{N}}
\newcommand{\Z}{\mathbb{Z}}
\newcommand{\Q}{\mathbb{Q}}
\newcommand{\R}{\mathbb{R}}
\newcommand{\C}{\mathbb{C}}

\newenvironment{problem}[2][Problem]{\begin{trivlist}
		\item[\hskip \labelsep {\bfseries #1}\hskip \labelsep {\bfseries #2.}]}{\end{trivlist}}

\newenvironment{answer}[2][Answer]{\begin{trivlist}
		\item[\hskip \labelsep {\bfseries #1}\hskip \labelsep {\bfseries #2.}]}{\end{trivlist}}

\newenvironment{theorem}[2][Theorem]{\begin{trivlist}
		\item[\hskip \labelsep {\bfseries #1}\hskip \labelsep {\bfseries #2.}]}{\end{trivlist}}

\newtheorem{thm}{Theorem}
\newtheorem{defn}{Definition}
\newtheorem{conv}{Convention}
\newtheorem{rem}{Remark}
\newtheorem{lem}{Lemma}
\newtheorem{cor}{Corollary}

\begin{document}
	
	\title{Exercises from Dummit and Foote Chapter 15 on Commutative Rings and Algebraic Geometry}
	\author{Wesley Basener}
	\maketitle
	
	\begin{problem}{1.1}
		Prove the converse to Hilbert's Basis Theorem: if the polynomial ring $R[x]$ is Noetherian,then $R$ is Noetherian.
		\begin{proof}
			Suppose $R[x]$ is Noetherian. Then any ideal in $I \subseteq R$ is also in $R[x]$. So $I$ must have finite generators $f_1, f_2, ..., f_n$ in $R[x]$. It remains to be seen that these generators can be strictly contained in $R$.
			
			Let $\alpha = a_0 +a_1x + ... + a_nx^n$ be any element of $I$. Then, $\alpha$ can be expressed by $\alpha = g_1f_1 + ... + g_nf_n$. If $\alpha$ is strictly in $R$, then setting all variables to $0$ is the identity. Hence $\alpha(0) = \alpha$, which lends $g_1(0)f_1(0) + ... + g_n(0)f_n(0) = \alpha$. Since $f_1(0), g_1(0), ..., f_n(0), g_n(0) \in R$, we have shown that any element of $I \cap R$ can be expressed by finite generators. Hence, $I$ is finitely generated in $R$. 
		\end{proof}
	\end{problem}
	
	\begin{problem}{1.2}
		Show that each of the following rings are not Noetherian by exhibiting an explicit infinite increasing chain of ideals:
		\begin{itemize}
			\item[\textbf{(a)}] the ring of continuous real valued functions on $[0, 1]$,
			\item[\textbf{(b)}] the ring of all functions from any infinite set $X$ to $\Z/2\Z$.
		\end{itemize}
		 \begin{proof}{(a)}
		 	Let $I_n$ be the ideal generated by functions which are $0$ on the interval $[1/n, 1]$. Clearly $I_1 \subseteq I_2 \subseteq I_3 \subseteq ...$ is an infinite chain. Since $I_n$ contains functions which are nonzero on $[1/(n+1),1/n]$, $I_n \not = I_{n+1}$. So each inclusion id proper an the chain is infinitely increasing.
		 \end{proof}
		 \begin{proof}[(b)]
		 	Using the AOC, let $x_1, x_2, x_3, ...$ be an ordered infinite subset of $X$. Let $I_n$ be the ideal generated by all elements $\sigma(x_i) = 0$ for $i \leq n$. Then, $I_1 \subseteq I_2 \subseteq I_3 \subseteq ...$ is an infinite chain. Since $I_n$ contains functions which send $x_{n+1}$ to $1$, the inclusions are again proper and the chain is increasing.
		 \end{proof}
	\end{problem}
	
	\begin{problem}{1.3}
		Prove that the field $k(x)$ of rational functions over $k$ in the variable $x$ is not a finitely generated $k$-algebra. (Recall that $k(x)$ is the field of fractions of the polynomial ring $k[x]$. Note that $k(x)$ is a finitely generated field extension over $k$.)
		\begin{proof}
			content...
		\end{proof}
	\end{problem}

\end{document}
\documentclass[10pt]{article}

\usepackage[margin=1in]{geometry} 
\usepackage{amsmath,amsthm,amssymb, graphicx, multicol, array}

\newcommand{\N}{\mathbb{N}}
\newcommand{\Z}{\mathbb{Z}}

\newenvironment{problem}[2][Problem]{\begin{trivlist}
		\item[\hskip \labelsep {\bfseries #1}\hskip \labelsep {\bfseries #2.}]}{\end{trivlist}}

\begin{document}
	
	\title{Exercises from Chapter 7}
	\author{Wesley Basener}
	\maketitle
	
	\begin{problem}{1.1}
		Show that $-1^{2}=1$ in $R$.
		\begin{proof}
			$-1 + (-1)^{2} = -1(1 + -1) = -1(0) = 0$
		\end{proof}
	\end{problem}
	
	\begin{problem}{1.2}
		Prove that if $u$ is a unit, then so is $-u$.
		\begin{proof}
			Let $w$ be the multiplicative inverse of $u$ and consider $-u \cdot w \cdot w$, by problem 1, when we multiply this by $-u$, we get $(-u)^{2} \cdot w^{2} = u \cdot w = 1$. Thus, $-u$ is a unit.
		\end{proof}
	\end{problem}	
	
	\begin{problem}{1.3}
		Let $R$ be a ring with identity and $S$ be a subring of $R$ containing the identity. Prove that if $u$ is a unit in $S$, then it is a unit in $R$. Show by example that the converse is false. 
		\begin{proof}
			If $u$ is a unit in $S$, then there is some $v$ in $S$, such that $uv=1$. Since $S$ is a subset of $R$, $v$ is also in $R$. So, $u$ is a unit in $R$. For the second part, consider the ring of rational numbers $\mathbb{Q}$, which has the integers $\mathbb{Z}$ as a subring. Although $2$ is a unit in $\mathbb{Q}$, with inverse $1/2$, it is not a unit in $\mathbb{Z}$.
		\end{proof}
	\end{problem}
	
	\begin{problem}
		Prove that the intersection of any nonempty set of subrings is also a subring.
		\begin{proof}
			Let $S_1, S_2, ...$ be a set of subrings in $R$ and let $S= S_1 \cap S_2 \cap ...$ be the intersection of all the subrings. Let $a,b,c$ be elements of $S$. Since $a$ and $b$ are in all $S_i$ in our set, $a \cdot b$ and $a+b$ will also be in every $S_i$ in our set, and will hence be in $S$. Thus, $S$ is closed on both binary operators. Now, since each $S_i$ is an additive group, the additive inverse of $a$ will be in each $S_i$.
		\end{proof}
	\end{problem}
	
\end{document}